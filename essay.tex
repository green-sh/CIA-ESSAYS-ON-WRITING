\documentclass[
    oneside,
    11pt,
]{memoir}
\usepackage{calc}
\usepackage{multicol}
% \usepackage[ocr-a]{ocr}
\usepackage{tgschola}
\usepackage[T1]{fontenc}

\usepackage{graphicx} % Required for inserting images
\usepackage[english]{babel}
\usepackage[pdfusetitle,colorlinks=false]{hyperref}
\usepackage[autostyle]{csquotes}
\usepackage[]{microtype} % better font rendering
\usepackage{float} % so fixing tables to a position
\usepackage{booktabs}
\usepackage{enumitem}
\usepackage{censor}
\usepackage{enumitem}
\usepackage[]{todonotes}

% Chapter Style
\renewcommand{\thechapter}{\Roman{chapter}}
\makechapterstyle{myarticle}{%
\chapterstyle{default}
  \setlength{\beforechapskip}{0pt}
  \renewcommand*{\chapterheadstart}{\vspace{\beforechapskip}}
  \setlength{\afterchapskip}{11pt}
  \renewcommand{\printchaptername}{\centering \thetitle\\ \vspace{22pt}}
  \renewcommand{\chapternamenum}{}
  \renewcommand{\chaptitlefont}{}
  \renewcommand{\chapnumfont}{}
  \renewcommand{\printchapternum}{\chapnumfont\thechapter. }
  \renewcommand{\afterchapternum}{}
  \renewcommand{\printchaptertitle}[1]{%
    \centering%
    ##1%
    }
  }

% title page
\setlength{\droptitle}{170pt}
\pretitle{\begin{center}}
\posttitle{\par\end{center}}
\predate{\begin{center}}
\postdate{\par\end{center}}

% \setsecnumdepth{chapter}
\chapterstyle{myarticle}
\pagestyle{plain}
\setlength{\parskip}{11pt}
\setlength{\cftchapternumwidth}{2.8em}% Width of \section numbers in ToC
\renewcommand{\cftchapterpresnum}{\hfill}% Inserted before \section numbers in ToC
\renewcommand{\cftchapteraftersnum}{. }

\title{Essays on CIA Writing}
\author{}
\date{AUGUST 1962}
\begin{document}

\frontmatter

\maketitle

\newpage

This is a typeset version, with metadata, chapters and links. I found this old document and typed it down. You can find \href{https://www.cia.gov/readingroom/document/cia-rdp78-00915r001400200001-3}{original version here.}

\mainmatter{}
\tableofcontents*
\thispagestyle{empty}

\newpage

\chapter{Introduction: What is Wrong}
It is the purpose of these few essays to look at some of the practices of CIA writing, to relate them to writing in general, especially to the best writing, and to suggest improvements. Most of the things this first essay says the later essays will develop but there is not time to say all things fully; some things can only be said and then left for the reader to develop in his own mind and in his own way. And the topics the essays cover do not exhaust CIA writing; the writer has taken up those topics that seem most to need attention or about which he has something worth saying. A good deal of writing, in CIA and outside, perhaps one-fourth to one-half, needs no comment and deserves none, it is neither good nor bad; it is more or less satisfactory: But there is very little distinguished writing anywhere, and what these essays are interested in is the practices that prevent distinguished, or even good writing in CIA.

Is there such a thing as CIA writing? In the sense that we do a lot of writing, there is, clearly enough. But what about the sense of distinctive writing --- is there CIA writing with qualities of its own? Probably so, although CIA writing shares many of its qualities with other governmental writing and even with writing in general in the United States today. Are the distinctive aspects of CIA writing good or bad? They are bad. So too are the aspects that are not distinctive, the aspects that are true of most governmental writing. Are the aspects intended or accidental? Some are the one and some are the other but all together the aspects seem accidental because there is no intention wide enough or constant enough to be worth the name; the attempts to write well-whatever the result-are few and occur here and there in CIA with almost no relation to the other parts of the agency. There are several submerged intentions, however, that are widely felt; these are less explicit than implicit and they are almost below the level of consciousness. They come from speeds cautions and a vague sense of professionalism.

There is a peculiar relation between intelligence in the sense of information secretly obtained and the writing that expresses that intelligence or comments on it or uses it. If the intelligence is not expressed in words or other symbols, it does not exist for us. A plan, an intentions a decision may exist in the mind of a foreign official but until we have that decision in words we do not know it. We know intelligence only in the words that express it. In one sense the intelligence and the words do not exist apart from each other; they are one and the same thing. This relation between intelligence information and the words that express it is no different in essence from the relation between any idea or happening and the words that express it, but it is peculiar because intelligence information is critical or obtained at some risk. If the words bearing it are inadequate, a great deal of effort may be injured or wasted and many harmful consequences may follow. With some frequency the understanding of an intelligence information report from a station abroad depends upon one word or phrase; if only, we say, the case officer had said clearly what must have been in his mind, if only he had anticipated what questions we would ask.

In whatever way we assess or state this close relation between information and the writing that contains it, we all have some feeling and respect for it. The wonder is that we do not have more, or, perhaps, that our respect does not produce a better result, for the evidence is that a good deal of CIA writing is unsatisfactory. It talks constantly about clarifying (a fond cliche) this or that but it lacks one of the elements of clarity---terseness. And although it deals, by occupation, with information that is, by its own assertion, vital (another cliche) to the United States, the vitality is enfeebled by poorly chosen or unnecessary words. Even in cables where long words and extra words cost money, there is often a fullness, sometimes a richness, that is at war with both economy and clarity. This overwriting, this excess of verbiage, this hodgepodge of jargon, cliches, and euphemisms, is the worst sin of CIA writing and is almost entirely verbal-not grammatical at all in the meaning of syntax, inflection, and spelling. Another sin, related and significant, is the riotous capitalization and the profligacy with space that seem, on the one hand, to result from the same apathy to shortness, precision, and clearness that produces the wordiness, and, on the other, to result from an imitation of the practices of advertising. We may yet see an QCI daily or weekly summary in which one sentence of intelligence information is placed all alone in the center of one long page like the diminutive Volkswagen centered in the vastness of white space, with the caption not \enquote{Think small,} but \enquote{Look big.}

To The effect of all this profusion---in words, in capitals, in space--- is just the opposite of inviting the interest of an intelligent reader; it repels him. It seems intended to attract, or, if not to attract, to serve, the person who cannot read or has no time to read. Intentionally or not it insults an intelligent reader because it assumes he cannot read without an abundance of paragraphing, a lavish use of space, and several times the number of words needed. He is given no credit. On the one hand, nothing is left for him to decide; he is not supposed to know what words mean; he is not supposed to take affront or to take the wrong meaning. He is not expected to keep an idea in mind through five, eight, or ten sentences of a paragraph but is spoonfed a sentence or two at a time; On the other hand, he is expected to penetrate the plethora of words and come out with a meaning, to jump his eyes from one hill to another over the valleys of space, to have his eyes knocked out with surname after surname in solid capitals and many common nouns improperly capitalized, and to digest a thick repast of would-be professional language and not be sick.

Such writing does not expect an intelligent reader, The truth seems to be that it does not expect any kind of reader, and this is one reason why the bad writers write this bad way: they have no sense of audience or they have a very wrong sense-they are writing for themselves or others like themselves. This is likely true and leads us to the next question. What can we do about this kind of writing? We must first realize what this writing is, recognize it for what it is. At the same time we must realize that such writing is written for others who write the same way---it is a fraternity of professional writers who sanction each others writing. Once we have recognized what is wrong with CIA writing, the next step, of improving it, will follow almost inevitably from the first step; if we start looking at what we have written, start tasting it, savoring it (some of this is bound to be uncomfortable), and then go on to finding what is wrong, we shall, almost willi---nilly, start writing better. There are two things, however, that will be necessary. One is the understanding that there is an art to writing and that speaking English as our mother tongue does not by itself qualify us to writeo The other is the intolerance of supervisors for bad writing. As long as the next higher person accepts bad writing, he will get ito When he recognizes, or learns to recognize, some of the differences between good and bad writing and then refuses to accept the bad., one-half of the battle will be over.

\chapter{Anyone Can Write}
Before we look further at CIA writing to find out what is wrong with it and what to do about it, we must deal with the widespread, though usually unexpressed, belief that anyone can write. This belief seems to be as common in this agency as is the assumption in the army that anyone can teach. Yes, anyone can write, and anyone can teach, in a way, in a sense. But really to write, or to teach, is something else, and being born and raised in an English-speaking country does not itself qualify us to write well. None of us would argue that anyone can design a house, install an electrical system, teach advanced algebra, or fire an eighty-one millimeter mortar. Why, then, do we assume that a person can write for publication, for others to use, can write professionally, without any professional training? Harold Ross, the editor for many years of the New Yorker, did not assume this. James Thurber tells us that when he interviewed Ross for a job, Ross asked, \enquote{Do you know English?} and when Thurber replied that he thought he knew English, Ross replied, \enquote{Everybody thinks he knows English, but nobody does.}

Many of us refuse to admit that any particular training or skill is necessary. We say, \enquote{I don't care how it is written, only what it says.} This is a critical statement, for the answer to it is that how it is written and what it says are inseparable, they are the same thing. An example of this indifference or confusion is the number of persons in CIA who talk about \enquote{substantive things,} meaning operational things, in the narrow, DDP sense, as over against the often unexpressed but clearly intended reportorial or non-operational things. This attitude, almost contempt, towards what \enquote{substance} means (how can such persons make any sense out of \enquote{and, like this insubstantial pageant faded, leave not a rack behind}?) is a part of the larger attitude towards the relation of form and content, for to such persons content has substance but form does not and of course we need concern ourselves with only substance.

The idea that a person is not necessarily qualified to write English for reason of having been born in this country and of having graduated from high school or college is intended here not to eliminate from our writers those who fail to meet some standard of competence (there is of course no chance yet of applying any such standard) but to induce our writers, at whatever level, to recognize that writing is an art and needs cultivation. If writing is of no importance, if only the substance of an intelligence information report counts, if it makes no difference how a directive or a dispatch is written as long as it has got the facts in it, then anyone can compose our reports, directives, and dispatches, as indeed anyone today does. But if writing is important, if content and form are related, if it makes a difference how a report or anything else is written,then we must recog- nize, admit this.

Learning to write is like learning any other art---it cannot be taught only by a textbook or a workshop. It is much more a matter of inclination, of a personts bent. If a reports editor or an analyst is an amateur of writing, he will learn to write with or without a textbook. No one can be an amateur, let alone a professional, of writing until he sees that writing is an art, that it is not something that all of us do equally well because we are all literate, that it requires interest or concern, first, and, next, practice, coaching, training. Any intelligent, college-educated, or even high-school educated, person can learn to write well for CIA purposes who wishes to learn to write well. And if he does not wish, no number of textbooks or workshops can teach him. And he cannot wish by an act of will alone, there must be some spark, there must be at least a little passion. And there must also be some encouragement.

These essays are intended for such persons, for persons who have some idea of the difference between good and bad writing, who have some wish to write well, who get some pleasure out of a well-written sentence; intended for such persons and for their supervisors. Anyone can edit an intelligence information report or write a directive to be read by hundreds of persons without batting an eye or opening one. But eventually we shall require our writers to be as professional as our case officers (on operations) or our TSS men. It is still a curious thought that we should require our writers to be able to write, to be interested in improving their writing, to be able to discuss writing. These essays are meant to make that thought less curious.

\chapter{Taste, Authority, Prescription}
Any question of style, whether it is one of selecting a word, spacing sentences, or putting a summary first instead of last, is finally a question of taste or authority or prescription. Taste means someone's personal decision; authority means some commonly accepted judgment such as a dictionary or a book on usage; and prescription means an instruction. Authority and taste are the same except that authority is more widely accepted than taste, but any authority that does not assert prescription is arguable, for only prescription, although it is based upon someone's taste in choosing among authorities, has decided and has issued an instruction: you will do it this way. In the absence of prescription anyone may assert his taste and invoke authority, just as anyone may quote the Bible to support his point. The recent controversy over the revised edition of \emph{Websterts New International Dictionary} by the G. and C Merriam Company centered in this question of taste, authorit and prescription. The critics of the dictionary condemned it as avoiding decisions of taste and of being tasteless; they wanted the dictionary to be an authority. The Merriam company answered that making such decisions would be prescription and that the job of a dictionary is to record and not to prescribe. The fact that both sides were right is an example of what this essay is trying to say: in matters of style the final judge is taste. In the lack of a national academy of American English, anyone may speak or write as he pleases. Usually he will be controlled by some prescriptions, whether of a publishing house or a governmental office, and usually he will pay a good deal of attention to usage as expressed in one authority or another; but prescription is often not thoroughgoing and it is often questionable, since the prescribers too must make personal decisions; and, of courses, authority, unless it is backed up by prescription, can be appealed.

The purpose of this series of essays on writing is to present some principles about writing for the consideration of those writers who want to take a second thought about their writing. Nothing here is prescriptive unless someone chooses to make it so, but some quarrels cannot be avoided where prescription has already acted. The authority for the essays is good English usage as practised by the best writers and as discussed in H. W. Fowler's \emph{A Dictionary of Modern English Usage} (to name one authority and the best) and the writer's own taste. Since choosing the best writers and deciding what English usage is good and what is bad are personal decisions, one cannot avoid one's own taste. A question of style is always, finally, a question of taste unless someone has prescribed what we are to do. This license is, of course, more seeming than actual, for we are all bound by conventions---of spelling, of punctuation, of sentence structure, and of diction. What these essays try to do is to give the principles or sense behind the conventions, to make decisions when conventions conflict or offer a choice, and to repeal conventions that have worn themselves out.

The underlying idea of the essays is that prescription without explanation is unsatisfactory: persons who earn their living by writing, in whole or in part, need to get to the bottom of things, to look at their writing professionally. Prescription by itself is easy for many of us because it does not require us to understand. We write this way or that because we are told to, and this is enough. Thus our typists all skip two spaces after typing the paragraph number and the period because the typing schools taught them so, and none of them is interested in why two instead of one, why not threes why, indeed, not six. This cosmetic attitude about space needs to be corrected by the attitude that asks why. Why, for example, do we skip a space between words and is one space enough? All these questions are a part of the writerts job. He should leave nothing to chance and nothing to typists. Based upon whatever authority he chooses and obedient to whatever prescription there is, he exercises his own taste, his own decision, where he can.

\chapter{Jargon or Less Is More}
The chief and the worst aspect of CIA writing is the failure to let words say what they have to say, to use simple words and let them alone. The result of this failure is a thick paste of words, a conglomeration of jargon, cliches, and euphemisms, of redundancies, pomposities, and irrelevancies, that instead of accomplishing more accomplishes less. The dominant characteristic is too much. This is true of words and of everything elseo It is true of space, which we use lavishly. Apparently a study that takes up twice as many pages as it needs is twice as well written or impressive. It is true of the subjects or titles of our dispatches and information reports. We have given up the old idea that a subject should be as short as possible and consist of a noun and a few modifiers. Now we try to say in our subject all that the dispatch or report has to say. (00-B-3,228,697,26 June 1962, has a subject of ten lines containing eleven units separated by slashes.) It is true of our fondness for itemization, of breaking up a sentence of two or three lines into a sentence of one line or so followed by three or four items of a few words each and indented as sub-paragraphs and double spaced. It is true of our passion for numbering the items as though we were writing for imbeciles who cannot count up to three or keep three items separate. Thus, we would not write \enquote{Please go to the store and get a loaf of Brea d,a quart of milk, and a few apples,} but \enquote{Please go to the store and get (1) a loaf of bread, (2) a quart of milk, and (3) a few apples.} It is true most of all of words. In this refusal to use space modestly, to itemize only when doing so is needed, to number only reasonably, and to let words say what they have to say, CIA writing is like bad architecture. There are lines going every which way, a wall that serves only as a shell is thick enough to withstand naval bombardment, space is eccentric, and you open what you take to be a pantry door and fall into the basement. The answer in writing is Mies van der Rohe's answer in architecture-less is more. If we let the skin and bones show in our writing, we shall attain what Mies van der Rohe attained in his architecture: \enquote{He has eliminated so much that seems irrelevant that what is left stands forth with unexpected significance.}

CIA writing, in refusing to let words say what they mean, is no different from the writing in the other governmental agencies or in the rest of the country. It is a refusal common to writing in the United States today. The difference that concerns us is that in government the refusal has flowered as it has in no other place save among professional educators, who, like the professional jargoneers in government, are able to speak in words that seem intended, to produce not understanding but a vague assent or euphoria.

The refusal to let words alone, to let words say what they can say, to use simple and fresh words, has several names but the best name, the one that includes most of the worst aspects, is jargon. In the basic sense jargon means unintelligible words, gibberish. In the narrow sense jargon means the unintelligibility resulting from the special vocabulary of a business or profession-cant. In the broad sense jargon means the unintelligibility resulting not from true business, professional, or technical words but from a thickness, a circumlocution, a redundancy. This is the meaning Sir Arthur Quiller-Couch gave the Ord in his famous essay \enquote{Interlude. On Jargon} in \emph{On the Art of Writing} and is the meaning intended here. Quiller-Couch said that the two parents of jargon are caution and indolence. Thus out of the one or the other the minister in the House of Commons does not say \enquote{no} but \enquote{the answer is in the negative,} and thus we in CIA take refuge in the passive voice and write \enquote{it is believed that!} instead of \enquote{we believe that.} The two main vices of jargon, according to Quiller-Couch, are circumlocution and wooly abstract nouns instead of concrete ones. Both show in the chestnut \enquote{he expired in indigent circumstances} for \enquote{he died poor,} and in \enquote{natal anniversary} for \enquote{birthday.} And both are evident in this letter from the Veterans Administration: \enquote{Reference is made to your letter of September 14, 1954, relative to your insurance. The remittance of \$195.00 tendered November 4, 1.953, has now been associated with your account\dots It is regretted that you were advised that your insurance had lapsed.} Four passive voices, \enquote{relative to} for \enquote{about,} \enquote{advise} for \enquote{tell} and the two masterful strokes \enquote{tendered} and \enquote{associated} may bring a laugh to any reader, but a CIA reader who laughs had better cover up first the papers on his desk, for such circumlocutions and pomposities are common with us. The opposite of jargon is the simple or plain word, the active voice, the terse sentence. Here less is more; words are allowed to say what they have to say; and the result is clearer and stronger English than the cautious, fuzzy, roundabout, and plethoric English that is jargon or governmentalese or gobbledygooke

\chapter{The Right Word and Literature}
A chapter in Barrie's \emph{Sentimental Tommy} tells of Tommy's losing out in a contest because he bogged down in the middle of his essay trying to think of a single Scotch word for the number of people in a church.\ \enquote{Fuckle} was too few, \enquote{manzy} meant a swarm and the kirk was not buzzing, \enquote{mask} would mean the kirk was crammed, \enquote{flow} was not enough, and neither was \enquote{curran.} \enquote{Middling full} was accurate but Tommy wanted a single word. Only after the contest was over and lost did Tommy recollect the right word, the word he wanted; it was \enquote{hantle.}

Tommy's devotion to the right word at the expense of a prize may dismay those of us who have a cable or dispatch deadline to meet, but it will take some of that devotion to force us loose from our addiction to pompous or wornout or slovenly or imprecise words. We write about \enquote{prior planning} and warn a station not to take action without the \enquote{prior approval} of headquarters when obviously \enquote{planning} and \enquote{approval} are enough, and similarly we write about \enquote{preconditions} and about a case officers meeting his agent by \enquote{prearrangement.} Adding an unneeded \enquote{prior} or \enquote{pre} is one of the many ways we have of refusing to let words alone, or refusing to let them speak for themselves. This is the commonest and the worst of all our sins against simple, clear English. Sometimes we go a little further and seem to have no idea of what a word means, as when we use \enquote{vis-a-vis} not to mean \enquote{face-to-face} but \enquote{on}--\enquote{What are the station's views vis-a-vis the new Chinese Communist embassy?} But even this kind of illiteracy results less from plain ignorance than from the desire to look elegant, to talk professional. We are less interested in the right-word than in the pretentious or pseudo-professional word. This desire seems to be responsible for the virtual disappearance of \enquote{on} in cables; no case officer worth his salt uses anything but \enquote{re} (\enquote{instructions re liquidation})-we may not have mastered English but we have mastered the professional lingo and if \enquote{re} does not prove it we shall toss in \enquote{modus operandi} or \enquote{caveat.} Whatever the cause---ignorance, would-be professionalism, caution, indolence, or something else---such writing is bad writing and shows no devotion at all to the right word. It ignores also the warning attributed to Mark Twain that the difference between the right word and the almost right word is like the difference between lightning and the lightning bug.

Some readers are sure to object at this point that CIA is not writing literature and that all this talk about less is more and the right word is literary talk and irrelevant. But this idea is as false as the idea that anyone can write because his native language is English or he has been to school. In writing the only example we have that is worth following is the writing of the best writers. And writing of whatever kind that achieves excellence is literature or near to it. We cannot learn to improve our day~by-day, chiefly expository writing in CIA unless we apply the lessons of the expert writers to our own tasks. And the difference between what we call creative writing and the writing that we do is not entirely, perhaps not largely, a difference in kind but in degree: The hack architect will not learn from another hack; he may learn from Mies or Wright or Le Corbusier: In one sense all of us who write in CIA are hacks but the less hack writing we do for CIA the better it will be for CIA and also, surely, for ourselves. We need the example of the best writers, and we need to write as well as we can.

A discouraging thing about the writers in CIA is that many of us seem to have no professional interest in our writing: Several signs show this. One is the lack of discussion about writing: There is a good deal of shop-talk about how to conduct an operation, about whether the reports from Source-1 are any good, about the Sino-Soviet conflict, and about the health of Mao Tse-tung. But there is little talk about how to write a contact report, an intelligence information report, a dispatch, or a study: This sign is a part of the belief that anyone can write and also of the belief that form and content are different things and that if you take care of the content the form will take care of itself: Another sign is the lack of interest in literature: as writers we do not read other writers, especially the creative writers; or, if we do read them, we do not learn from them, there is no application to our own writing. We do not examine even such writings as the New York \emph{Times} and the \emph{New Yorker}. We read that newspaper and that magazine and others, but we do not read them critically, observantly, for the benefit of our writing: If we have read James Ageels \emph{A Death in the Famiily}, we have liked it but we see no relation between that novel and our writingo And yet, surely, persons who earn their living in whole or in large part by writing cannot be said to take their writing seriously, professionally, if they do not do what any baseball player does, study the hitting and fielding of the other and better players for tips, for learning.

One thing other writers can teach us is to get something out of words: We can get something out of words by letting them alone. This is not what the person did who wrote \enquote{Mr.\ X showed that he was visibly disturbed.} Usually \enquote{show} does not accompany \enquote{visibly,} as in the sentence \enquote{McKechnie [ at Cooperstown ] was visibly choked with emotion.} Whether it does or not, \enquote{visibly} is unnecessary: The writer will not trust the reader to understand that when a reporter observes that the baseball old-timer was choked with emotion the choking was visible. To say \enquote{the showed he was visibly disturbed} commits two redundancies instead of one. It is flabby writing. Letting words say what they can say is only one part of the job but it is basic also to the other parts.

Another part is using words close to their etymology. Thus, if we learn from Fowler or our own dictionaries that the root of \enquote{meticulous} means to fear (\enquote{the walked along the parapet meticulously}), we won't use the word to mean \enquote{culate,} \enquote{scrupulous,} or \enquote{punctilious,} as Graham Greene does on the first page of \emph{The Quiet American} (\enquote{he was very meticulous about small courtesies}). If we are going to let words say what they can say and discriminate among words, we must pay some attention to original meanings. Another example is \enquote{paramilitary.} The Greek prefix \enquote{para} means beside, alongside of, beyond; the Latin prefix \enquote{para} comes from the word \enquote{to defend.} Thus a parachute is a defense against falling but a paramilitary unit is one that is beside the regular military unit and thus an irregular unit. The Freikorps troops in Germany in 1919 were paramilitary troops, existing alongside of, or outside of, the regular military troops. Today, however, a paramilitary outfit is likely to be understood popularly as a special but nevertheless regular outfit. We may choose to use \enquote{paramilitary} in the original sense or in a later sense but whichever we choose we should choose knowingly. Similarly \enquote{hallmark,} meaning originally the mark made on gold or silver articles at Goldsmith's Hall in London as a sign of purity, is scarcely apt as a figure of speech when we mean \enquote{badge} or \enquote{criterions.} And, finally, we might let go of \enquote{expedite} if we learned that it meant originally to free someone caught by his foot.

We may, in the third place, in order to get terseness or freshness into our writing, use old words anew. This may be as modest as returning to the simple words being replaced by pretentious words. This is almost a rediscovery of a word's meaning, as it is with the more complex word \enquote{meticulous.} Thus we may use \enquote{say,} \enquote{tell,}, or \enquote{answer} instead of \enquote{advise} or \enquote{apprise} when we make a statement or request one, and we may save \enquote{advise} for statements giving advice and \enquote{apprise} for a learned sense of \enquote{inform.} Or this use of old words anew may mean returning to an earlier, less common, almost forgotten meaning, \enquote{Inspiration} in the sense of encouragement has almost displaced the earlier, more literal meaning of breathing (still retained in calisthenics) but we may call upon that earlier meaning at will, even combining the two meanings (\enquote{Standing next to Ulbricht as he spoke, Khrushchev inspired the words on Stalin's excesses}), The same is true of \enquote{speculate} in the sense of to contemplate or to ponder a subject. Or we may pick up a word that few persons ever use any more and freshen our writing with it.\ \enquote{Factitious} is such a word.

In the fourth place we may make new words out of old ones, put together new phrases, \enquote{Carpetbagger,} \enquote{chain gang,} \enquote{back talk,} \enquote{gerryrrmander,} \enquote{pussyfooter,} \enquote{skyscraper,} and \enquote{brain talk} are exampios of thio. So too are the more recent \enquote{iron curtains} \enquote{bamboo screon} and \enquote{crypto Communist.} We need this fourth way and even the other three less than the poets, novelists, and journalists need them, but we cannot be indifferent to them all of the time simply because we write exposition or argument instead of poems, fiction, or journalism. And the poets, novelists, and journalists are there to excite us with the right words new or old.

Some CIA writers object to this kind of talk, saying, \enquote{I am not a writer. I am a researcher (or a reports officer or a case officer or an analyst) first and a writer only second and incidentally.} This is a defensive, amazing, and wrong-headed remark, for the research, the report, the analysis appear only in the writings The remark is as though Faulkner had asserted that he was first of all, a researcher into the annals of Yoknapatawpha County and, only in the second place, a writer. Anyone can write a novel about Oxford, Mississippi, or any other place in the United States or elsewhere; and anyone can write an account of the 22nd CPSU Congress; and, of course, the novelist must get his people straight and the analyst must get his facts straight; but, in both the novel and the study, the writing is as important as the people or the facts. Indeed, it is more important since the thing that distinguishes either the novel or the analysis is the way it is written, is the style, is the presentations. Take a building instead of a novel or a study. Any architect and likely any contractor can design and see built a three bedroom, two bath residence in Bethesda with or without gables, with or without shutters, with or without a central hallway, or an office building for States GSA, or CIA, but who will enjoy looking at this house or building or living or working in it? It will do, as most of our writing does. But four walls, some rooms and a roof become good, meaningful, enjoyable, rewarding, excellent only to the extent that Wright, Mies, Le Corbusier, or some other architect worth the name has designed them. The same is true of the novel and the study.

Our writing in CIA is as far from Faulkner or Agee as our splitlevel in Bethesda is from Wright's Robie House---true. But let us read Faulkner and Agee and all the other kinds of writings between them and us, and let us show some effect in our own writing of the reading, If we do we shall write better for CIA. There is no reason for CIA writing to be painful to read, no reason save bad writing. To avoid jargon, to find the right word, to accomplish more by seeming to say less is not the whole story but it is a large part. And other and better writers will help us here.

\chapter{Farewell \enquote{Tell!} or Abondened Words and the New Jargon}

These essays have already suggested that the parents identified by Sir Arthur Quller-Couch as the producers of jargon, namely, caution and indolence, are not enough to account for the jargon in CIA. To those parents we must add a, third,  would-be professionalism, and probably a fourth, pretentiousness, although it is arguable that whenever we try to sound professional we fall into pretentiousness and that whenever we are pretentious it is a vague sense of professionalism that stirs use. Whether the reason is caution, indolence, professionalism, pretentiousness, or something else. CIA writers have embraced a whole new list of long, fuzzy, and elegant words and, in doing so, have killed off certain older and simpler words. These deaths would excite no grief at all if the reason for the killing were not suspect. But suspect it is and worse---the reason is guilty, for it seems to be a part of the refusal to let words speak for themselves, of the contempt for simplicity, and of the preference for the long word over the short, the high sounding over the low sounding, the euphemism over the plain word, the pretentious over the humble, and the ambiguous over the certain. Thus we prefer \enquote{component} to \enquote{part} and when we have hit our stride we write, in elegant redundancy, \enquote{component parts.} We do not ask a station chief abroad to answer us but to advise us, and we do not tell him of an operational lead but apprise him. Iongfellow in CIA would have written, \enquote{Apprise me not in mournful numbers.} Occasionally the less talented among us suffer a little confusion and appraise the station chief of what we want to tell him.

Nothing comes out or appears, not even information in the press, it emerges, like Venus Anodyomene from the waves. We do not start or begin a plan, a project, or an operation; we initiate it. Like royalty on its way from Winchester to Windsor we and our plans do not go ahead but proceed. And we have more designs in our heads than a devil or an architect, for our projects do not intend to accomplish an objective but are designed to do so (\enquote{x is organizing a political movement designed to counter\dots}.) An informant does not say or assert that a certain hostile embassy is subsidizing students but claims it, and the station abroad does not send the informant's report home but forwards or transmits it. At home our files do not show that the informant has a dubious background but indicate it or reflect it, and the several traces that show this are not many but multiple, and the members of the embassy concerned are not persons but individuals or personalities, and they are not ordinary members of the embassy but ranking members. We do not ask the station to report regularly or from time to time on the embassy but to report on a continuing basis. In this kind of writing no one asks a question but poses it; we do not talk about someone's stand or position on an issue but about his posture (\enquote{the PKI was angered by Sulkarno's posture on the Indonesian-Chinese Communist dispute}); and all cores are hard, never more so than when the cores are Comsnunist Leaders are not simply leaders, they are top leaders. And if we do not call them top leaders, then we prefer \enquote{leadership.} One writer, not CIA, has achieved the combination \enquote{top cored.} We refuse to say that now Mr.\ X is the prime minister and also the legal adviser; we insist on saying that currently he is the one and concurrently the other. We have just about killed \enquote{before} and \enquote{after;} for some reason \enquote{prior} and \enquote{subsequently} have the right professional flavor (\enquote{we plan to see him subsequently to the publication of his four books}).\ \enquote{More} is another short word that we cannot bring ourselves to used. In the sentence \enquote{the Viet Cong have mounted increasingly successful ambushes\dots.} it is impossible to decide what \enquote{increasingly} means---each ambush more successful than the preceding one so that the first ambush of all was some ten degrees less successful than the tenth, or the last few ambushes more successful than the earlier ones? Whatever it means, how does it differ from \enquote{more}? \enquote{Increasingly,}\enquote{intensify,} and \enquote{progressively} have become professional cliches with many of use \enquote{Coordinate} is now so common and so loose that there is no longer any hope that we shall ever again use \enquote{concur,} \enquote{approve,} or \enquote{accept.} and we have given up the three-letter word \enquote{use} for the unctuous \enquote{utilize.} The holds of these new professional words is so strong and the appeal of the shorter, older, simpler words is so weak that if anyone dared to ask a station to answer at once he would certainly find someone up the line changing his request to \enquote{Advise soonest.} This last word is a good example of the deterioration in thought and language, for \enquote{soonest} has come to mean \enquote{when you get around to it. }

Similarly \enquote{speed} and \enquote{speed up} have fallen before \enquote{accelerate} and \enquote{hurry} before \enquote{expedite.} The fact that all haste has gone out of expedition does not bother the jargoneers among us anymore than does the fact that we talk about first  priority objectives and second priority objectives. Probably this most abused word of all is \enquote{potential.} We do not ask what an agent can do for us but what his potential, security leaks are not known or possible but known or potential, and the \enquote{potential utility of this questionnaire can be illustrated by\dots.}

We are not content with a few old standbys so familiar and so convenient that to stop using them would mean to stop breathing (\enquote{coordinate,} \enquote{expedites,} \enquote{accelerate}), but in our zeal to make our writing sound, pretentious and professional we are adopting new ones. One of these is \enquote{viable} and the day is coming close when none of us will assert that an economy will live or endure or that a plan will work but that they are all viable. Another recent addition is \enquote{counter-productive} and a wayward attempt to exploit the Sino-Soviet conflict would not fail or rebound or hurt us---it would be counter-productive. The latest newcomer, used so far only by our first-class jargoneerss, is \enquote{escalate.} This is not something you do to get from the first floor in Woody's to the second, but what owners of weapons systems do when they raise the ante.

A list of some of these examples of jargons with the dead or dying words in parentheses, follows.

\noindent
\begin{multicols}{2}
	\noindent
	advise (tell, answer, inform)  \\
	apprise (tell)  \\
	approximately (about)  \\
	appraise (grade, rate)  \\
	accelerate (speed up)  \\
	claim (say, assert)  \\
	component (part)  \\
	concerning (on)  \\
	concurrently (also, at the same time)  \\
	continuing basis (regularly) \\
	coordinate (approve, accept)  \\
	currently (now)  \\
	design (intend)  \\
	emerge (come out)  \\
	expedite (hurry)  \\
	forward (send)  \\
	increasingly (more)  \\
	indicate (show)  \\
	individual (person) \\
	initiate (start) \\
	intensify (increase) \\
	leadership (leaders) \\
	limited (small) \\
	launch (begin)  \\
	multiple (many)  \\
	pose (put)  \\
	posture (position)  \\
	potential (strengths, possible) \\
	prior (before) \\
	priority (order, importance)  \\
	proceed (go)  \\
	reflect (show, say)  \\
	requirements (needs)  \\
	role (part)  \\
	subsequently (after)  \\
	substantial (large)  \\
	transmit, transmittal (send-ing) \\
	utilize (use) \\
	viable (workable, practical) \\
\end{multicols}

\chapter{More Jargon}
This essay is a companion to the preceding essays \enquote{Farewell, \enquote{Tell} or Abandoned Words and the New Jargon.} The difference between the two lists of words is that the words in the earlier list are replacing older and simpler words (\enquote{multiple} for \enquote{many} and \enquote{advise} for \enquote{tell}) whereas the words in this list are used incorrectly or needlessly, or are over used and, so have become meaningless cliches. Some words in this list are on their way to replacing simpler words; others have already replaced words or phrases that we can scarcely recall. Whatever the differences, all the words in the two lists have one thing in common, they are jargon because they are unnecessary or pretentious or fuzzy or would-be professionals they are smooth and rounded and have no cutting edge. They are not peculiar to us but our use of them is peculiar to our work-they have the caution, sonorousness, ambiguity, or gloss that is stamp and unction to us. Anyone can write \enquote{speed up}, we write \enquote{accelerate} (earlier list). And only we have the excuse of duty to.write \enquote{infiltration} (this list) whether there is any filtering going on or not.

What shall we do with these words? The best thing is to stop using them. What shall we do instead? Replace them with something simpler, something less ambitious, something a little more particular. This will take thought, and effort, for these words are familiar and convenient to use We shall feel tongue-tied for the first few times we try to avoid them. But if we think of what we mean, of what we are trying to say, if we push our thoughts, demand that we know in detail, in the furthest sense, what we want to say, we shall find good alternatives. Take \enquote{across the board.} The metaphor in this word is dead to almost all of us, and because it is most of us use the expression vaguely. It is a good rule to drop all figures of speech as soon as we stop thinking of the figure in them, of the comparison they originally called to mind. In Mr.\ Mitchellts sentences below, \enquote{everywhere} will do, but we cannot be sure what word to use because we cannot be sure what Mr.\ Mitchell meant-in all jobs? in all governmental agencies? in all crafts and industries? We do not know; \enquote{across the board} does not tell use.

The evil of these words is that the reader reads them as glibly as the writer writes them; there is little thought, little appreciation, by either. When we use many of these words in a memo, dispatch, or regulation, the reader scarcely understands what he reads without study and guessing. An example from work different from ours and yet somewhat similar to some of the work that we produce is this paragraph from \enquote{The Parameters of Social Movements: \emph{A Formal Paradigm},} which Daniel. Bell wrote as a parody, as a hoax, but which several of his learned colleages took seriously. Why shouldn't they? They had been reading such stuff for a long time.\ \enquote{The purpose of this scheme is to present a to nomic dichotomization which would allow for unilinear comparisons. In this fashion we could hope to distinguish the relevant variables which determine the functional specificities of social movements, Any classificatory scheme is, essentially, an answer to some implicit other schemed In this instance, it is an attempt to answer the various hylozoic theories which deny that social categories can be separable.}

It is possible to write several equally convincing paragraphs of our own by a liberal use of \enquote{across the board,} \enquote{cognizant,} \enquote{frame of reference,} \enquote{on a continuing basis,} \enquote{implementation,} \enquote{modus operandi,} and almost any other words in the preceding list or the following one. It would not hurt, of course, to get in a few uses of \enquote{strategy} aand \enquote{tactics.}

\textbf{across the board.} \enquote{We must improve individual competence, present and prospective, across the board\dots.} (James P.\ Mitchell, former secretary of Labor, in the Washington Post, 7 Jebruary 1960).

\textbf{allegedly.} The original meaning is declaration as if under oath or affirmation without proof. This is too serious or heavy for us. We should use \enquote{say.}

\textbf{automatically} \enquote{monthly progress reports\dots are automatically routed through the CE Section.} The word means self acting. Usually, as here, the context denies the sense of the word; the reports are routed by schedule or custom or they are routed regularly.

\textbf{close colaboration.} \enquote{The field worked in close collaboration with commo.} The field worked with commo, or worked closely with commo.

\textbf{cognizant.} \enquote{We have nothing on record of a derogatory nature re- garding Source-1 and are cognizant that what information we do possess is far from complete.} We have nothing derogatory on Sourced and what we have is far from complete.

\textbf{comparatively.} The attempts on the KPD to transform their huge \enquote{organization into an illegal movement were so clumsy that Hitler found it comparatively easy to ban the part\dots.} Omit \enquote{comparatively.} Words of size, difficulty, and so forth involve comparisons or relations in their meanings; there is always abase or standard involved. Thus \enquote{he is tall} means in comparison with the rest of use. The only time we need \enquote{comparatively} is when we have a particular base in mind and express it: \enquote{in comparison to the efforts of Ebert, Hitler found it easy to ban\dots.}

\textbf{consolidate,} \enquote{The outstanding feature of Soviet internal policies\dots has been\dots the further consolidation of Khrushchev's position.} \enquote{Pro-Communists appear to be making steady progress in their program of consolidating strength in Singapore's labor movement and in Chinese schools\dots.} (OCI, weekly summary, 19 May 1960). What does \enquote{consolidate} mean-to make solid? If so, what about further consolidation? \enquote{Consolidation} is as wornout as \enquote{coordinate.} In the first sentence, use \enquote{strengthening}; in the second use \enquote{winning.}

\textbf{core.} \enquote{Each commune is run by a hard core of Communists\dots.} \enquote{Manifestly the linkage of Mao and Stalin destroys a core plank in the original \enquote{Maoist} thesis} (the China Quarterly, October-December 1960). If \enquote{core} is used accurately, it does not need \enquote{hard.} In the first sentence omit \enquote{hard}; in the second change \enquote{core} to \enquote{basic.}

\textbf{counterproductive} \enquote{Believe any attempt fabricate latter info for psych exploitation would be counterproductive} (Dir-21052, 17 January 1962). Would fail or hurt use.

\textbf{escalate} The USA does not \enquote{want to jump into Laos in any such obvious manner as would get the Soviet Union back up and force it to escalate its existing military aid to Pathet Lao rebels} (Washington Post, 27 March 1961). Use \enquote{increase} and omit \enquote{existing.}

\textbf{frame of reference.} Think of the figure of speech in \enquote{frame of reference.} The original meaning of \enquote{term} is limit; terms of reference define the scope of an inquiry. It is better to avoid these phrases than to use them glibly. It is best to say what we mean simply.

\textbf{functional.} \enquote{X collates information functionally on a continuing basis on all facets of activities of\dots.} It is impossible to decide what the word means here; the word has become almost a meaningless counter for rounding out a phrase or adding a little flavor.

\textbf{GOI.} \enquote{Sometime before 16 May 1961 USSR privately assured government of India that Soviets will support India in border difficulties with Communist China} This expression, whether written out or as initials, has become an affectation. Here \enquote{India} is right by itself just as \enquote{USSR} is. It has long been an accepted trope to use a part for a whole or a whole for a part, and in \enquote{the United States signed the charter of the United Nations Organization} the meaning, of course, is the United States government. Is it only time until we get \enquote{GOUS}?

\textbf{implement.} Fulfill, execute, effect, put into actions See \enquote{in terms of,} below.

\textbf{infiltration} \enquote{So-and-So infiltrated legally to Canton by train on 26 November.} \enquote{X was infiltrated into Indonesia by smuggling methods.} If the figure of speech does not apply (it does not in the first sentence) or if it is expressed in another word (it is in the second sentence), we should avoid this word. In the first sentence, \enquote{So-and So entered Canton legally by train}; in the second, \enquote{X was smuggled into Indonesia.}

\textbf{instrument.} \enquote{The Related Missions Directive (RMD) and the Annual Assessment of Progress Report constitute the basic management instruments by which\dots.} \enquote{Constitute the basic management instruments} is a rich, full-bodied, sonorous expression but what does it mean? Only that the directive and report are the means by which etc.

\textbf{literal, bilateral, unilateral.} Often unnecessary. If we mean onesided why not say so? \enquote{Unilateral operations} means our own operations.

\textbf{local leadership.} (see the last page.) \enquote{In 1958 the Chinese Communists asked Communist leader in local latin American countries\dots.} Omit.\ \enquote{Local} is almost a pure example of our using words for ritual and not for meaning. We use it when we do not need it.

\textbf{marginal.} \enquote{Information of more than marginal value} and \enquote{only of marginal interestst.} Use \enquote{little.}

\textbf{maximize.} \enquote{In order to maximize savings and investment for future growth.} \enquote{Ten to twenty is the outside maximum.} In the first sentence use \enquote{increase}; in the second use \enquote{limit.}

\textbf{modus operandi.} This Latin expression is no more professional than the English. Method of operating, operational methods.

\textbf{mutual.} \enquote{The U.S.A. and Cuba should hold each other in mutual esteemm.} \enquote{Mutual} is unnecessary. See Fowler, \emph{Modern English Usage}, p.368.

\textbf{over-all.} \enquote{The increase in over-all industrial production is\dots} (OCI, weekly summary, 21 July 196O). Omit (as understood) or use \enquote{total.}

\textbf{per se.} \enquote{Our estimate of his security per se is adequate. However, he might become compromised through other operations to which he is linked.} If he might become compromised, this is a part of his security. Better, \enquote{He himself is secure enough but he might\dots.}

\textbf{personnel.} \enquote{Following is personnel assessment of Mr.\ So-and-So.} Omit.

\textbf{pinpoint.} \enquote{It is believed that X would have no trouble in pinpointing Agent Z as the source of the information.} This word long ago lost its force from overuse and misuse. Here it is misused for \enquote{identify.}

\textbf{precondition.} \enquote{Russia, when the Soviets came to power, had more favorable preconditions for industrialization than exist in Asia today.} No \enquote{pre.}

\textbf{pressure.} Use \enquote{press} for the verb. Use something else for the noun-weight, force, compulsion.

\textbf{progressively.} \enquote{As the time of the outbreak of World War II drew nears ISH activity was focused progressively on espionage and sabotage.} Omit as meaningless or use \enquote{more and more.}

\textbf{ranking.} \enquote{X met General Y upon latter's request be put in touch with ranking Z official} \enquote{Like every other Soviet leader, he Suslov was\dots.} (OCI, weekly 1 28 July 1960). In the first sentence replace \enquote{ranking} with \enquote{top}; in the second omit it because it is redundant with \enquote{leader.}

\textbf{re.} \enquote{Instructions re liquidation.} On. (In \enquote{Re\censor{eferences}} means reference.)

\textbf{relatively.} Unnecessary; omit it. See \enquote{comparatively,} above.

\textbf{reportedly.} \enquote{Mao Tae-tung reportedly told an American journalist last autumn that the average Chinese peasant currently was lucky to receive 1.000 calories in food a day} (OCI, daily digest, 1 May 1961). This word has become a great convenience but it is seldom needed or apt. Many statements in a newspaper article or an information report or an analysis of news are the statements of someone or other, and unless we qualify every statement we should not qualify only one or two. The \enquote{reportedly} here is likely telling us that the American journalist, said that Mao made the statement but that there is no confirmation from Mao. But this goes without saying, doesn't it? If we drop out \enquote{reportedly} we get the same meaning. Like much else in our writing \enquote{reportedly} is extra, it is interference.

\textbf{represents} \enquote{It is of special interest to us because it represents an indication of the PPC role in performing a FCC function on Taiwan.} \enquote{Source-1 has represented one of the most likely channels through which\dots.} More often than not misused, as here, because no representation is involved. Replace with \enquote{is} in the first sentence and \enquote{has been} in the second.

\textbf{sector} \enquote{The problems in the agricultural sector\dots were manifold and serious} (Esau II). Padding. Say, simply, \enquote{in agriculture.}

\textbf{substantive} \enquote{Substantive issues.} \enquote{This fusion of operational substantive information is an expression of our view\dots.} \censor{AAAAAAAAAAAAAAAAAA} \enquote{Substantive} means independent or substantial; operational may or may not be substantive. Some parts of our agency, in contrast to the sentence quoted here, use \enquote{substantive} to mean operational. It is best to throw this word away.

\textbf{in terms of.} \enquote{In my experience it has been relatively easy to implement the border agreement in terms of the population of the areas which changed hands\dots.} \enquote{In terms of his impact on the young the most powerful figure in modern Chinese letters} (the China Quarterly, December 1960). In the first sentence use \enquote{on}; in the second use \enquote{in.}

\textbf{vis-a-vis.} \enquote{We had several discussions vis-a-vis those subjects.} \enquote{Phoumi also believes that Captain Kong Le' is having second thoughts about Kong Le's position vis-a-vis the Communists.} Use \enquote{on.}

\textbf{leadership.} \enquote{Communist influence on X is considerable  especially at the leadership level.} Especially among the leaders.

\chapter{Coherence in Composition}

The arrangement of the words in a sentence, of the sentences in a paragraph, and of the paragraphs in a composition-the relation of the parts to the whole-depends upon the material and the intention of the writers. The intention of the writer includes his idea of the reader. This is true whether the composition is a poem, a novel, a play, an essays, an editorial, a report, or an analysis. There are some rules or conventions but these allow a good deal of variaty. There is, for example, the convention that any piece of writing must have a beginning, a middle, and an end. But the beginning need not begin with the first thing (in chronological or logical order) but in the middle. Thus the \emph{Odysse} does not open with the beginning of Ulysses' ten years of adventure but with the last years, and the does not begin with the start of the Trojan War but with the end; and the earlier action is taken care of after the two epics have got under way. In stories, of course, the writer is interested in suspense, climax, and denouement. But these things are the concerns not of the story writer alone but of almost all other writers, creative and otherwise, in some way or another. They are less the concern of the expository writer, than of the fiction writer but they come into all writing. In other words, each writer arranges his material to produce the effect that he wants. At the same time the material has something to say about this, for some material is easy to handle one way but hard another, even impossible. And there is more than one ways, arrangements, of securing the same effect.

When our material is little and our intention simple, we may write, \enquote{Dear Sir: Please send me one ream of your bond paper number seven,} or \enquote{We enclose two copies of our comment on the study you sent us, \enquote{or} Here lies John Doe.} But when there is more than a little to say and there are several ways to say it and our intention becomes complex, we have problems? Shall we say at the outset what we are doing and why? Or wait until we have done something? Shall we run through the house fast, one room after the other, and then go over the house again, slowly and carefully, or shall we walk up to the front door and take each room as it comes once and for all? You come home from work tired, and at dinner your wife starts telling you about the visit of the inspector from the Chesapeake and Potomac Gas Light Company, who came out to set the thermostat. You are impatient and want to hear the end first-did he or did he not succeed in correcting the fault? Your wife, who has had a weary day of another kind, without all the quickening and profitable talks and interruptions that you have had, wants to tell her story in her own way, which is one step forward and two steps backward. This is a good way to tell a story, but there are other ways; and readers will not agree on which is best?

There is no right way to tell such a story or to write an essays There are some ways more conventional (customary) than others; there are some ways more startling (less customary) than others; there are always new ways, or new combinations of old waysp to find. The writer decides according to his material and his intention just as an architect designs a house according to the lot, the needs of the clientp materials, climate, and so forth. Two persons asked to design a theater for the same lot will produce different plans. So two several writers on the same subject. There is, therefore, no one way to write a paper on evaluating the reports of Source-l9 or on the development of the All Asian Trade Union Conference, or on the tensions between the CPSU and the CPC. There are usually several, ways, even a dozen ways. And the reader or audience is only one part of the problem and probably less important than the material and our own minds. If someone orders us to underline all the important names in red so that he can find his way along the route or print the important phrases in capital letters so that he can tell what is important, or just to put down the conclusions and omit the argument we may obey and write this way. But usually there are no such instructions and usually there is no one reader or even one group of readers to write fors but several readers or several groups. Shakespeare may have put the porter scene into r>beth for the sake of the groundlings9 but did he believe that the gentlemen would not laugh too, and for whom did he write the sleepwalking scene?

The emphasis here that the material and our intention decide the arrangement is like the slogan in architecture that form follows function. The architect will design a building according to the job it has to do and the materials he has to build with. The writer will look for an arrangement inherent in the material or in a combination of the material and his intentions.

There are two basic kinds of arrangement-the logical and the artificial. If we arrange according to logic, the logic may be chronological, based on time; or geographic, based on space. Or the logic may be based on size (we move from the small to the large) or on complexity (we move from the simple to the complex). Or the logic may be based on association, with one thing or point suggesting another. Or the logic may be based on importance, and we move from the less important to the more, These logical ways of arrangement are the common ones for most of the writing that we doa In the artificial or fabricated kind of arrangement we impose an arrangement chosen for some emotional or aesthetic effect upon one or more of the logical ways. Thus the \emph{Iliad} and many other epics begin in the middle of things and then go ahead according to time but with interruptions to get in the earlier action. The flashback is a part of this technique. In fiction Conrad's \emph{Lord Jim} and much of Faulkner are examples of a complicated, even tortured, time schemed In this kind of writing time is broken up and arranged for the effect of belief, acceptance, by the readers.

The question in our writing of where to put the summary (first or last or in both places) is a question both of logic and our intent, the effect we want. Logically, if we begin at the beginning, the summary belongs last. But we may choose to summarize before we begin. The normal or most common place is last, but there is no rule. We may put the summary where we please. Whether, then, we begin at the beginning or in the middle or at the end depends upon us, our purpose, and what we have to write about. The only sure thing is that we must begin somewhere.

\chapter{Format}

Questions of format are questions of logic and appearance, of pleasing the eye and the mind without indulging or abusing either. Paragraphing, for example, should look logical to the eye and be logical to the mind, and the width of the left-hand and right-hand margins should be enough to make reading easy, to keep the eye from falling off the page, and not anything more. Excessive space is a waste of paper. As soon as we start to beguile our reader with more space than he needs, he will treat us as an advertisement; he will not read us but will start grabbing and skipping. Some of our pages present more whiteness than blackness, more space than typing. This is good for the paper companies, for the weight of our studies if we judge them by pounds, and for readers whom we expect not to read but to sample. It is bad for readers who are expected to read. Here the principle is the simplest, most sensible, most uncluttered page possible. The less extra space, the less underlining, the fewer quotation marks, the fewer indentions, the fewer blocks, the fewer headings, the better. Unless we engage in billboard writing we are not trying to shock or surprise the reader; we are trying to invite him. We want to enlist his mind; we want him to read us attentively and smoothly. We do not invite him to skip or to give up or to put down. We may help him to skip or to go ahead and then return; but we do not encourage him to do these. Here, then, in format, we follow a few basic things that we are used to, that are easy to understand, and that are pleasing to the eye.

The rule for headings is the fewer the bettor, the less capitalization the better, the less underlining the better, and the shorter the better. When a paper has only one heading and that is the title, only initial letters need be capitalized-it is conspicuous enough by the setting and spacing. We may, of course, capitalize all the letters of the title and we may then underline the title, but these devices are unnecessary; the reader cannot miss the title because it comes first, is set off by extra space, and is centered. Excluding the title, we may need one, two, or three levels of headings. The more complicated the paper, the more headings, the more aid we try to give the reader in going and coming, stopping and pondering, taking a breath and going on. It is a nice question at what point headings hinder instead of helping the reader. And it is possible to gain the benefits of headings and at the same time avoid cutting up our paper with them by relying on transitional expressions at the beginnings and endings of paragraphs. But some papers lend themselves to headings and are helped by them.

The way, one way, to handle several levels of headings is explained in \emph{A Manual for Writers of Dissertations} by Kate L. Turabian (the University of Chicago Preass). This manual is an example of how someone's personal taste becomes an authority. Although intended for the writers of dissertations at one university, the manual has been adopted by other universities and by some non-academic offices. A similar thing is true, in printings of the same university's \emph{A Manual of Style.} Miss Turabian recommends that if you have three levels of headings, in addition to the title of the paper, you capitalize and center the first level (chapter headings, for example), center the second level with capitalization of initial letters only, and underline and paragraph the third level. Usually we require only two levels of headings in addition to the title of the paper. If so, the second and third of the three levels given above will do. If we use these (a centered heading with initial letters capitalized for the first level and an underlined heading set into the paragraph for the second level), we have the choice of numbering paragraphs. If we number the paragraphs, each paragraph should have an underlined heading. If we do not number, then an underlined heading will apply to all the paragraphs below it until the next underlined heading occurs. With numbered paragraphs, whether we also use headings for them, we may use sub-paragraphs with letters.

Whether we number our paragraphs depends, as do headings, on the kind of paper we are writing and the impression we want to make. Usually numbering alone cannot hurt a paper and may help it. For one thing numbering the paragraphs helps in referring to the paper or discussing it. Numbering also is likely to help prevent the writer from turning out many short paragraphs; we are more likely to consider a paragraph traditionally (a topic sentence with some sentences of development) if we number it. Numbering, like indention, emphasizes the units of a composition. In a fairly long and complicated paper, numbering the paragraphs and using headings of one or two levels are likely to help the reader. His eye will skip the headings and the numbers if he wishes. This means that ordinarily the paragraphs should be written as though the headings and numbers did not exist, so that the composition can be read easily and intelligently without them. They are added and not essential.

There are two main questions about paragraphing. One is whether to use indented paragraphing or block paragraphing. Indented paragraphing is commoner, pleasing to the eye, and easy to follow. The other question is whether to indent the second and succeeding lines of a sub-paragraph to the right of the left-hand margin. The answer is no, There is no reason for moving sub-paragraphs to the right; this makes sense for outlines but not for compositions. So, indent each paragraph five spaces and carry al1 succeeding lines to the left-hand margin, and indent sub-paragraphs under the first word of the main paragraphs and carry all succeeding lines to the lefthand margin. Lists of items are something else. One easy and logical way to handle them is start the list where the paragraph starts and, if an item runs to more than one line, indent the second and succeeding lines three spaces to the right of the first line.

The use today of the slant line (or slash) instead of the hyphen has gone so far that a discussion here can hope only to check the slant, not return the hyphen. Two things make the slant suspicious on sight. One is that the slant is, riotous where other bad practices are riotous, in governmental writing. Another is that the slant is too conspicuous for what it does. Until governmental offices began using the slant in place of the hyphen, the chief uses were to separate lines of verse when run into the text, to replace periods in abbreviations (km/sec), and to separate corrections in proofreading. It is clear from these conventional uses that the slant meant separation only and did not imply a relation. This is Fowler's use in his book on usage; the slant separates sentences used as examples to show that the sentences are not related. The most frequent uses of the slant, then, have been in extraordinary writing where the need to be clear and at the same time to save space is great. There is, therefore, no need to use the slant in ordinary writing. In fact, since the slant is a kind of shorthand and needs interpretation, the use of it discourages reading and encourages skipping. There is no need to write CC/CPSU for CC of the CPSU unless we are writing a kind of shorthand throughout, any more than there is any need for writing \enquote{I went to the store and bought a bottle/milk.} Except for the conventional uses there is nothing the slant can do that the hyphen or \enquote{and} or some other short word or two cannot do. WH-Arg for the Argentina unit of the Western Hemisphere office is as clear as WH/Arg and better looking; the slant in DD/P is unnecessary; and \enquote{and} is clear where the slant is not in \enquote{he will be held for debriefing/ultimate disposition.} The question of shorthand is important for us. When we are pressed for space and engaged in unconventional writing, anything that will save space and at the same time be clear is welcome. But in ordinary writing we should write things out because this is the way the reader reads them and because this way gives pleasure. Some magazines, as a point of style, write out numbers for this reason, thus five hundred thousand dollars instead of \$500,000. If we choose the slant or any other device as a short way, along with other devices, there can be no objection- -this is a special kind of writing. But in ordinary writing to use the slant in preference to the hyphen or to avoid writing out \enquote{of the} is not sensibleo.

The most common use of the slant is in the expression \enquote{and/or} (\enquote{A and/or B}), which offers three alternatives (A, B, A and B). Despite this offer, there are two objections to it. One is that it is shorthand and is better written out when the three alternatives are really intended and need being made explicit (\enquote{you may take the candy or the cigar or both,} instead of \enquote{you may take the candy and/or cigar}) and the other is that it is often unnecessary. Here it is as bad as \enquote{if, as, and when.} Let us choose one of the three in \enquote{if, as, and when.}: it will do. Let us choose \enquote{and} or \enquote{or}; one will usually do. Thus in the statement that heredity or environment decides our character we do not rule out that both decide it, and it would be silly to write \enquote{heredity and/or environment.}

\chapter{Capitalization}

Capitalization is an interesting subject to write about because the underlying principles are clear yet the confusion is great and because almost no one feels sure that he knows what he is doing. Capitalization, even more than the other subjects these essays have treated, is bound up with taste, authority, and prescription, dealt with in an earlier essay, for in no other subject is there the great difference between principle and practice that there is in capitalization. In other words, taste and the authority of the press (newspapers and magazines), which is a combination of the taste of the press and the convention which that taste both follows and creates, have played loose with the basic principles. This is true also of the practices in CIA, for we exercise our taste in the face of other authority (or convention) and in the face of logic. Thus we capitalize \enquote{agency} when it refers to CIA and \enquote{bureau} when it refers to the FBI, and thus we capitalize \enquote{division} when it refers to one of the area divisions and are likely to extend this courtesy to other parts of CIA, including stations abroad. Another and different example of our taste is our capitalizing of titles when they do not stand before the name of the person. Thus we almost always write \enquote{He is the Chief of \censor{some department}} instead of \enquote{chief} although we would never write \enquote{She is the Clerk Typist of \censor{some other department.}} There is no written prescription about such capitalizings but we do them so often that apparently we feel the force of someone's will. Perhaps all we feel is the pull of prestige with the upper case elevating some words up out of the common run of lowercase nouns.

We are free, of course, to capitalize as we please, just as we are free to dress our sentences in all the jargon that we please. But the question is whether our taste is any good, whether we know what we are doing, and whether we have thought the matter out. Here, as in jargon or format or coherence, we must try to start with some principles. The most this essay can do is to propose a simple and clear set of rules with fidelity to a few principles but in violation of many of today's practices.

There are two principles to follow in capitalization. One is the sight of the page. This means the fewer capital letters the better; the less a page is broken up with capitals or anything else (slant marks, quotation marks, underlinings, and so on) the better. The reason is that the page both looks better to the eye and reads better to the mind or ear. We write in words and we ask our reader to read words. Unless we are writing an advertisement our emphasis is on words and not on marks or typography. This is the reason why some experimental writers have given up capitalization and even punctuation. They want their readers to \emph{read}, to join them in what they have written, created. They know there is a special pleasure when the reader reads the lines or sentences on his own, the way the writers have intended and with no aids other than the words themselves. Thus E. E. Cummings writes his own name and his verse without capitals and with the sparest punctuation:

\begin{displayquote}
next to of course god america
i love you land of the pilgrime' and so forth oh
say can you see by the dawn's early my
country `tis of centuries come and go
\end{displayquote}

The second principle is the basic convention in English, we capitalize only proper nouns and only in the full, exact name. The trouble comes from defining a proper noun and from handling short names, alternatives, and substitute names. By adding the first principle (the fewer capitals the better for the benefit of eye and ear) to the second (upper case for proper nouns only) we come out with a narrow definition of a proper noun and a rejection of short names (unless they are substitutes) and alternatives. 

A proper noun is one that distinguishes an individual, a particular, person or thing from others of the same class; it needs no limiting modifier. Thus, we capitalize \enquote{Jane} because it is the name of one of Mr.\ Brown's daughters but we do not capitalize \enquote{girl} or \enquote{student} even when these words refer to Jane.We say, meaning Jane, \enquote{This girl was a student at Michigan State University. After she graduated from the university. She was employed by CIA. She has worked for the agency since 1960.} Here we have almost all the examples we need. When we name the individual, the one singular and particular person or thing, with her name (Jane) or its name (Michigan State University or CIA), we capitalize the name; when we do not, we don't-we do not capitalize words that refer to or designate the individual person or thing, but only the words that \emph{name}. This is a simple rule, easy to understand and, easy to apply, Why don't we apply it? 

One trouble is the capitalization of common nouns when they are used as titles and precede the name of a person. Since we capitalize the title when it is a part of the name (\enquote{Jane's father, Major Brown, received a Purple Heart}), we are led into capitalizing it when it is not (\enquote{Jane's father was a Major during the war}), and we capitalize military ranks? especially, with fervor. But if we think of some less exalted and less martial titles (janitor, foreman, policeman, cook, shoe-shine boy) and try to imagine our capitalizing then (\enquote{Jake, the Janitor of our building, and his son Jim, the Shoe-Shine Boy in Mike's Barber Shop} all capitals are wrong save in Jake, Jim, and Mike), we shall get another view of what we are doing with military and other titles. The rule, of course, is to capitalize a title only when it precedes the name and thus is a part of the name. 

The magazine \emph{Time} has added to this problem of titles by writing titles of all kinds and lengths in front of the name instead of after.\ \enquote{Chairman of the State Economic Commission Pc I-po,} \enquote{University of California Economist Neil H. Jacoby.} and \enquote{Child Psychologist Anna Freud,} instead of \enquote{Po I-po, the chairman of the State Economic Commission,} \enquote{Neil H. Jacoby, an economist at the University of California,} and \enquote{Anna Freud, a child psychologist.} This has likely caused some of us to capitalize such titles regardless of where they occur. Some of us have also begun to imitate \emph{Time} and are writing such sentences as this: \enquote{The police twice arrested President of the Executive Committee and Vice President of the Ad Interim Standing Committee Malachi Jones.} We are out of breath and indifferent by the time we find out that-Jones is the person the police arrested; and there is no reason not to write \enquote{The police twice arrested Malachi Jones, the president\dots} save a kind of smartness or an attempt to make the person important, even portentous.

A minor point is that some of our writers are ignoring the difference between a title and an appositive by omitting the article before the appositive and thereby making the appositive a title yet without capitalizing the title as accompanying the name. In \enquote{Speaking at a Moscow disarmament conference on 9 July, chief Chinese delegate Mao Tun asserted\dots} we need a \enquote{the} before \enquote{chief} and a comma before and after \enquote{Mao Tun,} or capitalization of \enquote{chief Chinese delegate.} The use of the articles and the comma is, of course, better. This example serves to make another point, if we go on writing like this we shall change the language-we shall kill old conventions and beget new ones. The question in such killing and begetting is whether there is a gain. All of this has some bearing also on our rather silly and lazy habit of using the word \enquote{subject,} often with a capital, instead of the person's name when we are writing about an informant, an agent, or a person we have interviewed, and, in doing so, omitting the article before \enquote{subject,} so that we get a childish or halfliterate effects \enquote{Subject said\dots I told Subject\dots Subject reported that\dots.} 

When we put Mr.\ Kennedy's title before his name and write \enquote{The two senators who talked with President Kennedy were Senators Mansfield and Humphrey,} we are following an old convention and one that is a part, of the single basic principle in English capitalization- the use of capital letters for proper nouns, for the names, including titles, of individual persons or things. But when we do not put the title before the name there is no reason for capitalizing \enquote{president,} \enquote{senator,} or \enquote{director} anymore than there is for capitalizing \enquote{superintendent,} \enquote{teacher,} or \enquote{undertaker.} Thus the upper and lower cases in the following sentence are correct and according to the basic rules \enquote{The senior senator from Illinois asked the president whether he had discussed the matter with the director of the Central Intelligence Agency, President Kennedy said not with the director of that agency but with Mr.\ Hoover, the director of the Federal Bureau of Investigation.} The same thing is true of \enquote{prime minister.} 

There is no reason except courtesy, respect, or prestige. Most newspapers and magazines capitalize \enquote{president} when it refers to the president of the United States, and most CIA writings capitalize \enquote{director} when it refers to the director of CIA, for reasons apparently, of courtesy, respect, or prestige. But this is a poor reason because there is no way of telling where to stop. If we are going to decide capitalization of occupational or vocational labels according to degree, where are we going to draw the line? We can discriminate in favor of the president and the senators of the United States and against chairwomen, janitors, clerks, and even baseball managers and coaches, but what about the presidents of real-estate companies, commanders of armies, directors of funeral parlors, and pastors of churches? It seems sensible to stick to the basic distinction. A university dunning an alumnus for a contribution will capitalize the word \enquote{university} throughout the dun when it refers to his university but never when it refers to any other; and the advertising man will capitalize whatever words he pleases to impress his product on us. But there is no reason to impress us with the president of the United States or the director of the CIA; their offices are impressive enough. And the courtesy seems uncalled for, since it is misleading and allows or induces hundreds of other honorable capitals.

In addition to the problems of capitalizing titles there are the problems of alternatives for names, short names, and substitutes for names. When the word is a true substitute, capitalize it. Thus we capitalize \enquote{Dad} and \enquote{Mother} when they take the place of the name but not otherwise (\enquote{Jane asked Dad about vacation plans; she did not ask her mother}), and we capitalize \enquote{Old Hickory} as a substitute for Andrew Jackson. This last covers nicknames, including \enquote{Old Nick} for the Devil We do not, or should not, capitalize \enquote{administration} when it refers to the Kennedy administration. The word is not a substitute for President Kennedy or for the Cabinet; it is an alternative, a synonym, and is no different from referring to Casey Stengl, and his assistants as the coaching staff. The same thing goes for \enquote{government} when it means the government of the United States (the context will tell us) or of a state or city. The same thing goes also for \enquote{our nation's capital,} usually written \enquote{our Nation's Capital.} The rule here is not to capitalize alternative or short forms unless the short form amounts to a substitute. Thus we may capitalize \enquote{State} as a substitute for Department of State (but not \enquote{department} for Department of State unless we belong to State and like to pretend there is only one department) and \enquote{Army} as a substitute for the United States Army (in both, State and Army are the important words in the full name), but not \enquote{party} when referring to a certain Communist party (\enquote{Chou is a member of the Communist Party of China; he has been a member of the party since his student days in France}).

A knotty problem here is what to do with short forms of ministries and with the title \enquote{minister.} The title \enquote{minister} is no different from any other title (colonel, steward, salesman, case officer, typist) and should not be capitalized save when it is used as a part of the person's name. When we give the name of the ministry in full and accurately, we capitalize it. When we do not give it so, we may either capitalize it or not. We should capitalize it if we feel that the part we use stands for, substitutes for, the full name; we should not capitalize it if it is only an alternative. Thus we should write \enquote{Mr.\ X is an official of the Ministry of Education; he is the vice minister of education (Education). He used to be the minister of foreign affairs (Foreign Affairs).} The two ministries are the Ministry of Education and the Ministry of Foreign Affairs. When we use only \enquote{education} or \enquote{foreign affairs} we may use the lower case as simply referring to the work of the office or the upper case as a substitute for the full name. The phrase \enquote{the Minister of Welfare} is short for \enquote{the minister of the Ministry of Welfare} and the omission of \enquote{Ministry of} probably causes the capital in \enquote{Minister.} Here, then, \enquote{minister of welfare (Welfare)} is all right. 

There is another point we should recall here, a point that many of us have forgotten: we may, whenever we choose, treat the name of a \emph{thing} not as a name but as a description. Thus, if Mr.\ X is the chief (or head or minister) of the Bureau of External Affairs (this is the formal name) we may say \enquote{Mr.\ X is the chief of the Bureau of External Affairs} or \enquote{Mr.\ X is the chief of the bureau of external affairs.} We cannot do this with the names of persons, for the only alternative to the name of a person, outside of a nickname, is some other words since names of persons are not descriptions but unique and peculiar, almost arbitrary. Thus, for \enquote{Robert L. Brown} we must write \enquote{Jane's father} or \enquote{the major} or \enquote{my next-door neighbor} or the \enquote{winner of a Purple Heart.} When we know what the correct, formal name is, we should use it and capitalize it, but when we do not know or when we are in doubt, we need not capitalize. 

The capitalization of all the letters in surnames in dispatches is another matter. This is sensible if it is done for the benefit of clerks in files, but the practice has got out of hand and many writers or typists today are putting all surnames completely in capital letters, including the names of CIA employees, other Americans, and foreign officials like President de Gaulle. This widespread practice suggests that the complete capitalization of surnames serves no purpose save when it picks out the surname from several names, as in Chinese names. Two principles should apply here: how the page looks to the eye and how it reads to the ear and mind, and whether the capitalization makes carding easier. 

What this essay recommends, then, is that we use capitals only for names (proper nouns), for clear substitutes for names, and for titles written as a part of names; that we do not capitalize out of courtesy, respect, or prestige; and that we do not capitalize alternatives, references, or synonyms. When in doubt, don't capitalize: you are likely to be right if you do not capitalize and your page will look better with as few capitals as possible. 

The following is an outline summary of the points for handy references.
\begin{enumerate}
  \item The principles.

    a. The sight of the page and the reading of the page: the fewer capitals the better. When in doubt, no. This is like the fewer the better for other things: words, punctuation marks, quotation marks, underlining.

    b. The basic rule: only proper nouns and usually only in the exact named (A proper noun is one that distinguishes an individual, a particular, person or thing from others of the same class.)

  
    
    
  \item Alternatives for proper nouns: no. 
    
    a. He graduated from the University of Washington. He attended the university only three years.
    
    b. An address on Massachusetts Avenue; the avenue was\dots. 
    
    c. The Communist Party of China; a member for years of the Chinese party. 
    
    3. Short forms of a name: no, unless a substituted 
    
    a. He sent a letter to the Department of State. He never received an answer from State, or, he never received an answer from the department.
    
    b. He works for the National Police Agency. He was first employed by the agency in\dots. 
    
    c. You're in the Army now. (If you mean the U.S. Army.) 

    d. There were several hundred representatives at the Fifth National Congress of the World Federation of Clerical Workers. It was a crowded congress. Rafael Juston, general secretary of the federation, addressed the congress. Many of the delegates had attended the fourth congress but none the first. At the fifth congress there was less\dots. 

  \item Titles or job labels: no, unless used before the person's name.

    a. The French ambassador asked to see Ambassador Thompson. The ambassadors met at the American embassy.\ (\enquote{American embassy} is not the name of the embassy.) 

    b. The manager of the hotel said to the charwoman, \enquote{Now, Charwoman Mordaunt, what\dots.} 

    c. The coach of the Little League team\dots. I wrote to Coach Robinson of the Bethesda Blues, a Little League team.

    d. Please see the chief of Division X\dots. Are you in Chenk Brown's division? 

    e. He wrote to the director of the Nation's Capital Savings Company. The director answered\dots. 

    f. The advisers informed President Kennedy that\dots. The advisers left the president at five o'clock. The administration then issued a statement\dots. 

    g. Here is the information that the chief of the station requested. Please send a copy on to the \censor{name of station} Station.

    h. The Senate is one part of the Congress of the United States. Each state has two senators. Jane once called on Senator Douglas, but she missed seeing Mr.\ Dirksen, the other senator from Illinois. Kira Sam Rayburn was a member of the House of Represertives; he was a representative for a long time; he was the speaker of the House.

  \item Made-up or unofficial names: no. If you are in doubt whether a name is official or not, no. 

    a. The satellite countries, the Sino-Soviet bloo, the Communist conspiracy, the international Communist movement, the iron curtain. (The last may be capitalized for something like personification.) 

    b. The United States government, the government of the United States, the Burmese government, the government of India, the Russian government (Before the birth of \enquote{GOI} persons said India when they meant the government of India, a synecdoche of long standing.) 

    c. The United States consulate in Hong Kong, the American embassy in Karachi, the Indian embassy in Washington. (The official name of a USA embassy abroad is Embassy of the United States of America. We may choose to consider \enquote{American embassy,} or \enquote{United States embassy} as a substitute for the official name and capitalize each word, but it seems more sensible not to, for once we accept unofficial names for the official names we lose hold on what a name is-it is formal, official or nothing-and we do not bother to learn and use the official name.)
    \end{enumerate}

\chapter{The Passive Voise}

Next to jargon the worst aspect of CIA writing is the passive voice, When we use the passive voice needlessly we are, indeed, writing jargon since this voice is usually longer than the active and usually indirect, often cautious, and sometimes ambiguous. We seldom need it but we often use it because it assures and protects use Like \enquote{tends} (\enquote{the African National Congress tends to be Communistic}) the passive voice gives and it takes away-, it says whatever the writer decides, after the fact, he intended to say. There is a way out.\ \enquote{It is believed that Khrushchev will be succeeded by Brezhnev.} Who believes this? If the sentence is to make sense, all, persons, or virtually all persons, believe this. But if this is so, why not say it? Well, we can be caught up on that. But if not all persons, then who? Well, some experts on African Who are they? Well, uh. Finally, why don't you say you believe it? You do believe it, don't you? Y-e-e-s, but I can't, you know, insert myself; I can't appear, can I? Why not? You wrote it. So it goes. And not one passive voice in that short sentence but two. The first one comes from the caution that is one of the parents of jargon. The second one comes from the other parent, indolence, or, perhaps, like the first, from a vague feeling that \enquote{K will be succeeded by B} is less of an assertion than that \enquote{B will succeed K,} a feeling that we didn't quite say it.

There was no passive voice in Old English and Old English did not need it since Old English could say \enquote{this pleases me} or \enquote{I please myself} (instead of \enquote{I am pleased}). One historian of the English language has said: \enquote{It was long before the speaker was able to imagine an action without an object.} (The object in the active-voice sentence becomes the subject in the passive-voice sentence, and the subject of the active-voice sentence is suppressed or tied on with a \enquote{by}: \enquote{Khrushchev denounced Albania during the 22nd CPSU Congress.} \enquote{Albania was denounced during the 22nd CPSU Congress (by Khrushchev).}) The passive voice developed later by means of compounds (two or more verbs-one of them the intended verb and the other, an auxiliary, \enquote{become} or \enquote{to be}): \enquote{the book became written} turned into \enquote{the book was written,} and \enquote{the house was building}: (from \enquote{the house was on building,} \enquote{the house was a-building}) into \enquote{the house was being built.} None of this tells us whether or when to use the passive voice except to suggest that the development of the passive voice was a move towards complexity. Today using the passive voice is largely a question of style-of taste and intent. For the most part the passive voice is long, roundabout, hard to get hold of, and even ambiguous-. For uses of courtesy, caution, and indirection it is natural. Like much jargon, the passive voice sounds big, it sounds professional, it costs more, and it allows several ways out; it is safer than the active voiced. But it is no voice for us if we wish to write simply, clearly, and directly. 

There are some good or acceptable or unavoidable uses. One of these is when the actor (doer, effecter) is unknown (\enquote{that house was built in 1907}) or when the actor is impersonal (\enquote{words\dots which have been divested of their proper signification}--have been divested by history). Another is when the actor is unimportant or we prefer the emphasis on the action or result (the object in the active-voice sentence), not on the actor. The following three sentences illustrate this use: \enquote{Widespread malfeasance among local supervisors in China is disclosed in documents brought out by a defector} (contrast: A detector brought out documents that disclose\dots.); \enquote{Most American furniture is made in Grand Rapids, Michigan}; and \enquote{The deciding ballot was cast by the smallest state- Rhode Isand.} A third acceptable use is when the actor is general or diffused. In the sentence \enquote{At the same time, the Chinese have done nothing that can be considered a calculated step to aggravate tensions with Moscow.} we may accept \enquote{that can be considered a calculated step} as being generally true (\enquote{that can be considered a calculated step by anyone}) or prefer the active voice (\enquote{that we can consider a calculated step}); and in \enquote{Perhaps never before in peacetime has one of our Array divisions attracted as much attention as has been focused in recent weeks upon the 32nd Infantry Division,} the passive voice may be as pleasant as the active-voice alternative, \enquote{as much attention as the public has focused.}

The other uses of the passive voice are questionable or bad. The purpose of one of these uses is to avoid the appearance of egotism: the actor hides out of effacement or courtesy. (This often comes close to ambiguity or avoiding responsibility.) An example is \enquote{it was noted earlier,} where the writer avoids \enquote{I or we noted earlier.} Another questionable or bad use occurs when the writer does not quite know who the actor is and passes this ambiguity on to the reader. When we read the sentence \enquote{Whatever the Soviet intent may have been, the developments of 21- 23 October were probably unanticipated,} we ask unanticipated by whom? The third bad use seems to be an avoidance of responsibility or a passion for indirection. The criticisms of the sentence \enquote{It should be emphasized that the difference is only one of method} are two. First, the writer ought to state who should do the emphasizing. Second, the writer ought not to waste words saying the difference should be emphasized but should make the emphasis. A similar expression is \enquote{X (a person or a possibility) should not be underestimated.} This one is a cliche, a filler, a piece of padding that some of our writers use regularly in their writing. Usually there is no danger of anyone's underestimating X. The writer knows this but he shies away from saying either \enquote{I believe Communist China will invade Matsu} or \enquote{I do not believe Communist China will invade Matsu} and so writes \enquote{The possibility of Communist China's invading Matsu should not be underestimated.} Another example is \enquote{It is in this context that Khrushchev's ironic remark at the end of the congress must be viewed.} Must be viewed by whom? Let us say \enquote{It is in this context that I or we or you the reader must view K's ironic remark.} And another example of this avoidance of responsibility or passion for indirection is the sentence \enquote{We feel that although no immediate conclusions can be drawn\dots.} The writer has written \enquote{we feel} instead of \enquote{it is felt.} Good. But let him write also \enquote{although we cannot draw any conclusions.} 

A review of these uses of the passive voice suggests that we should never use it unless we do not know who the actor is or unless we consciously prefer it as rhetoric, for its effect. Nine times out of ten we use the passive voice out of indolence or caution; we are either too lazy to find the active-voice phrase or we are too cautious to risk it. There is no good reason to write \enquote{It is requested} to a field station or the FBI instead of \enquote{We (the division or CIA) request.}. The passive voice is infectious and it breaks out from an occasional use for politeness, impersonality, or caution and infects sentences that need to be only clear, direct, and straightforward. It is one of the cachets of bureaucracy, as the letter from the Veterans Administration, quoted in the fourth essay, shows. It is certainly commen in CIA. And since it is both common and often unwitting it will take two acts to get rid of it: recognition and will power. 

Writing in the active voice will produce clarity, directness, and sharpness. We have used the passive voice so often that we may feel uncomfortable with the clarity, directness, and sharpness of the active voice. There is no refuge with the active voice. The emperor is naked and everyone sees that he is naked. This is the way the sculptor wants the human body. This is the way the writer of exposition should want his prose.

\chapter{Writing, Editin, and Hack Writing}

Some of the trouble with CIA writing comes from a confusion of writing and editing and from what amounts to group writing. There is too much interference in our writing-there is too much rewriting by others and not enough by the original writers. The result is colorless, hack writing; no one is responsible. We should not have group writing. There are two sinners here. One, the original writer who does not care what anyone else does with his paper and takes little pride in it; the other, the person who rewrites it. Analyst A will write a study that Boss B (or Editor E) will edit so much that it ends up something that k cannot feel responsible for or take pride in; on the other hand, B is not responsible since all he has done is to revise A. Or Boss B will give A's study to three or four other analysts or editors or some combination of the two and then take a final turn at it himself. The production is a team product in form and content. It is a compromise or a potpourri or both. The team is responsible. This kind of writing works-it gets done, it is correct in facts and in grammar, it is cautious, it diffuses responsibility, it brings down no blame. But it has no distinction as writing and it gives no one who has taken part in it any pleasure in the writing. 

It is the lack of enjoyment in writing that is the worst aspect of hack writing. Enjoyment and responsibility go together and if we prepare data for someone else to write up or suffer the data we have written up to be rewritten by someone else, we cannot feel either enjoy ment or responsibility. Some editing and some rewriting are usually necessary but we commit more of each than we need to, and in the wrong ways because we do not take writing seriously enough. 

The job most of us have to do as writers calls for three things: imagination or ideas, facts or data, usually requiring some search, and writing. For the most part we are not divided into researchers and writers; for the most part the same person does both jobs. Thus the case officer on the desk looks up the data, finds out the answers, and writes the cable, memo, or dispatch (file traces done by persons specializing in them may be an exception); and thus the analyst keeps up with a certain subject and from time to time writes an analysis. Of the three things the ideas or insights are the most important and go hand in hand with the writing. The facts, though the largest element, are the least important- they are there for anyone to dig up, see, read, state. By the Titing, by the way the insights and facts are presented, or, by the way the imagination presents the facts, the facts will have a meaning, make an impact. Over and over again in our studies there is a chronological statement of what occured, sometimes with a summmary, also chronologically arranged, that is long enough to be a study in itself. This does not do much for use, The writer did not do much himself but look up the facts. He was unable to enlist his own imagination and so failed to enlist ours. Sometimes facts \emph{are} enough or are all that we can do, but just as often and likely more often there is something else to do if it amounts only to the way we write up present, arrange the facts. In some way we must impress ourselves on the facts, we must see the facts in some kind of light and then express this light. We can scarcely do this unless we are encouraged to do it. And we cannot be encouraged if the editing and rewriting are improperly handled. 

Editing and rewriting are not the same. The editor brings the paper into agreement with routine or mechanical things-spelling, punctuation, paragraphing, the conventions of the office or agency concerned. He enforces the style manual of the agency. (What style manual?) The less of this he does the better, and waywardness in these things-is better than the loss of flavor or originality. The editor also calls the writer's attention to any gaps, mistaken interpretations, and questionable judgments, and he suggests whatever improvements he pleases. (Some editors may not be capable of any of these.) But he returns the paper to the writer for the writer to revise. 

The rewriter goes beyond editing. He takes the paper of another as raw material or defective material and rewrites it in his own way. He fills in the gaps, corrects the interpretations, changes the judgments, and brushes up the diction, sentence structure, and grammar. If, then, the rewriter submits the rewritten paper to the original writer, there isn't much the original writer can do but accept it. The alternative of rewriting the rewritten paper is too much for him. This way of handling the job of getting a paper out is the wrong way for two reasons: it takes the pleasure out of writing and it takes the voice out of the paper-- such a paper speaks with no voice at all or with several voices; it has no timbre of its own.

(This essay does not deal with the editing or rewriting done by supervisors or others who are not capable of either. Some of this goes on, and it is disheartening to a writer to have a supervisor who knows less than he does about the subject and who can write less well make changes to assert his prerogative, to prove that he is a supervisor. This occurs in cables, dispatches, memos, analyses, and studies. It is bad. But it is less common than, and not as bad as the indifference of supervisors to the bad writing that they pass, and this is where the emphasis of these essays lies.) 

The insistence here that papers needing rewriting go back to the original writer for the rewriting may meet the argument that some writers are not capable of writing a satisfactory paper in the first place and not capable, therefore, of rewriting an unsatisfactory one. This is true. But the cure is not to rewrite the unsatisfactory paper of these writers but to get better writers. If a writer is not capable of improvement, it is wasteful to keep on letting him write and keep on rewriting him; it is also joyless. If a writer is capable of improvement, he will improve more by doing his own rewriting than by having someone else do it. He will also get some pleasure out of his work.

Perhaps this is the place to make the point, implicit in the other essays, that taking the pleasure out of writing-and this kind of editing and rewriting does more than that; it induces us to write in the same way and to stock everything we write with the same illiteracies, jargon, and, professional cliches-is not good for us either as an agency or persons. One of the evils of bureaucracy is just this group-think, group-write business. It is true that we have a problem. A study meant for outside use cannot be only Mr.\ A the writerte study; it must come from Mr.\ A's office and the agency. But, on the other hand, this does not mean that Mr.\ A's study must read as though it had been written by a machine which had rolled out all of Mr.\ A's ideas and mannerisms and at every other stroke had hit the levers \enquote{viable,} \enquote{counterproductive,} \enquote{consolidate,} \enquote{Subject,} \enquote{re,} \enquote{vis-a-vis,} \enquote{strategical,} \enquote{it is believed,} and \enquote{reportedly.} We live in houses of different designs and yet they are still houses, not the creation of an eccentric architect and likely to collapse at any moment, and none of us speaks well of the identical houses, row after row, that stand like cutouts on some of the nearby landscape. We need not write such identical prose or, if we do, speak well of it.

\chapter{A Miscellaney of Grammer and Style}

This essay takes up some miscellaneous points that bother some of our writers: Some of the points appear in the earlier essays but appear here in another context; other points are new. All of the points are fairly rudimentary and the treatment here is not novel or exceptional but is intended to refresh those writers who want to be aware of what they are doing. In many of the points there is the kind of jargon that the earlier essays have discussed. Jargon, in one form or another, is the worst trait in our writing. In some of the points there is simply a refusal to realize what words say: we write nonsense but do not see that we are doing so. The points add up to this: we need to look at everything in our writing --words, masks, grammar-- with a constantly clear eye but at words most of all. The arrangement is alphabetical, subjects first and then words. 

\begin{enumerate} 
  
\item \emph{Conjunctive adverb;} Position: not first. The following sentences are examples of the better place.

a. Five Japanese attended the meeting: There were \emph{also} two or three Burmese present.

b. The Orioles have won more games at home than they have lost. They have, \emph{moreover}, done well on the road.

c. The Korean Communist party, \emph{nevertheless}, sided with China at the Moscow conference. 

\item \emph{Connecting expeessions.} Get rid of them or make them simples 
  
  a.\ \enquote{Note that,} \enquote{it should be noted,} \enquote{it must be stated.}

  (1) \enquote{Despite these points of friction., it must be emphatically stated at the outset that the Sino-Soviet relationship continues to be marked by\dots.} (Problems of Communism, March April 1961). Omit the entire expression or say \enquote{we emphasize that\dots.} 
  
  b.\ \enquote{Should not be discounted,} \enquote{should not be underrated.} 
  
  (1) \enquote{\dots the likelihood that X is linked to the current Sino-Soviet dispute should not be discounted.} Better to say \enquote{X is likely linked to} or \enquote{X may be linked.} 
  
  (2) \enquote{As of 26 Dec. 1959, the possibility of such a meeting having taken place could not be discounted on the basis of information available in Bangkok or HK press circles} \censor{can also be} Rewritten: By 26 Dec 59 there was no information to confirm or deny the meeting.
  
  c.\ \enquote{In respect to,} \enquote{in reference to.} Use \enquote{on} (\enquote{X did not commit himself on the Sino-Soviet dispute}).

  d.\ \enquote{As of.} Use \enquote{by} or some other single word.
  
  e.\ \enquote{According to the available information.} This goes without saying. Cf \enquote{She is the most beautiful woman in the world, according to the available information.} 
  
  f.\ \enquote{As far as X is concerned.} We can usually shorten this to \enquote{in} or \enquote{on} or in some other way.
  
  (1) \enquote{Albania, as far as her relations with the USSR are concerned, will\dots.} Revised: Albania, in her relations with the USSR, will\dots. 
  
  (2) \enquote{As far as per-diem vouchers are concerned, the instruction requires that\dots.} Revised: the instruction on per-diem vouchers requires that\dots.
  
  \item \emph{Hyphens} with adverbs and participles, When the adverb and participle come after the noun they modify, they do not need a hyphen (a book strikingly designed; an employee well trained), When they come before the noun the only logical test is whether they modify the noun together, become, in effect, a unit, or the participle modifies the noun and the adverb modifies the participle. The trouble with this test is that we sometimes feel two words as a unit when they may not be so in sense. The trouble does not occur with adverbs ending in \enquote{ly.} The examples below are correct, are acceptable, 
  
  a. The queen was beautifully dressed, the most beautifully dressed woman of all the women at the receptiono 
  
  b. An exhaustively delayed game; a softly chanted song.
  
  c. A slow-moving wagon; a well-tanned face; an ill-timed remark; the below-quoted examples. (In each of these we may omit the hyphen with the argument that the adverb and participle are not, a unit.) 
  
  \item \emph{Initials for names.} Example: DPIK for Democratic Party of Iraqi Kurdistan. Use the full name or use a short form of the full name. Use the initials only when you refer to the organization frequently and, then put in the full name every page or so. It is unnecessary, though fixed in CIA, to follow the first use of the full name with the initials in a parenthesis. If the full name alone occurs in the first sentence and the initials in the second or third, any intelligent reader will solve the initials. Sometimes we put the initials after the full name and then do not use either name or initials again. 
  
  \item \emph{Noun on noun.} A new development (or, rather, a continuing development) in English, common inside and outside of CIA, is the use of a noun to modify a noun (\enquote{action program}) where the earlier way was to place the modifying noun after the modified noun by the use of \enquote{of} (\enquote{program of action}). There is nothing wrong here but some of the expressions are heavy, and the earlier way sometimes seems better-the more important noun comes first and the \enquote{of} phrase reads easily. Perhaps the chief question here is euphony-how the phrase sounds to us. The following examples show the new trend and, in parenthesis, an alternative.
  
  a. A sermon series (a series of sermons). 
  
  b. The coexistence policy (the policy of coexistence).
  
  c. Their main aid effort (their main effort in aid).
  
  d. Moscow's detente approach (Moscow's approach for a detente).
  
  e. Envelopment tactics (tactics of envelopment).
  
  f. Report of the peril points and probable moves in the cold war (report of the points of peril and\dots.).
  
  g. Significance of Soviet weapons developments (significance of the developments in Soviet weapons).

  \item \emph{Paraphrase.} copying and acknowledging, plagiarism.
  
  a. Ordinarily it is plagiarism if we copy more than a few words in succession from another writer without acknowledgement. If we do not want to quote, then we must paraphrase and also acknowledge. 
  
  b. Ordinarily we must acknowledge that we are using someone else's thoughts, writing. We may do this by quoting or by attribution with paraphrase. If we have made the thought of another person our thought and we do not want to bring him into our paper by quotation or paraphrases, then we must express the thought in our own words, here we must go further than paraphrasing the other writer. 
  
  c. When we are restating what someone else has said, using paraphrase in five or more sentences, we may state in our first sentence what we are doing and let this sentence control the rest of the paragraph. 

  \item \emph{Quotation marks}--when to quote a passage and when not.
  
  a. Do not quote two or three words of no extraordinary significance that can just as well stand without quotation marks. Example: The court found X's actions within the limits of \enquote{allowable criticism.} The writer of this sentence may argue that the court used the words \enquote{allowable criticism} and that he wants the exact words. The answer is that the reader will take those two words as the court's words and does not need the quotation marks to do so.
  
  b. To quote a word to question it is a recognized use of quotation marks, but a little of this goes a long way and some CIA papers are disfigured by the device. Two rules to follow are these: when you have quoted the word once, do not quote it again-the reader will understand that the question still applies; and do not quote it at all when the context makes it clear that you are using the word in someone else's meaning. This second rule applies to most of the Communist words and phrases that occur in our papers: proletarian internationalism, revisionism, people's democracy, national liberation.
  
  c. As a rule do not quote a long passage (three sentences or more) to illustrate a point. The reader will not read the passage (because it is our job to explain things and not to give the reader the undigested lumps) and, if he does, he may not agree that the passage illustrates what we say it does. Instead of quoting, paraphrase the passage, combine quoting and paraphrasing.
  
  d. In speech (\enquote{and I quote}) this is a part of not letting things alone; it is pretentious; it is unnecessary. We should show that we are quoting by inflection and pause. 
  
  e. Rely upon the context instead of upon quotation marks.
  
  \item \emph{Subiects and titles.}
  
  a. Confusion: we try to put the thesis or a summary of the whole paper into the subject, the title, and we confuse the subject with what we do to it.
  
  b. This is far from the old idea of a subject or title as a noun with a modifier or two (The Old Curiosity Shop, Hamlet, The Canterbury Tales, A Farewell to Arms, The Sound and the Fury, The Village Blacksmith, Henry Esmond). Today in CIA \enquote{A Farewell to Arms} would be \enquote{An account of a young American ambulance driver in the first world war who fe11 in love with an English girl and lost her when she died in childbirtho} 
  
  c. Examples.
  
  (1) \enquote{Comments on JCP document in Moscow Meeting.} The subject is not our comments; the subject is the JCP document. What we do is to comment on it. Omit \enquote{comments on.} 
  
  (2) \enquote{Transmission of a Request from SMOTH0} We are transmitting the request. The subject is the requests Omit \enquote{transmission of.} 
  
  (3) \censor{Part of the sentence} \enquote{Attitudes of Employees of the Foreign Languages Press (Peking) Toward Conditions in China and the November 1960 Moscow Declarations.} Revised: Attitudes in the Foreign Languages Press, Peking, 
  
  (4) \enquote{The Anti-Military Apparat of the Nambo Regional Committee of the Lao Dong Party in South Vietnam.} Revised: An Anti-Military Apparat in South Vietnam, or, The Lao Dong Party's Anti-Military Apparat in South Vietnam.
  
  d. Titles within subjects. A title that is a part of a subject must appear as a title or be changed to merge with the rest of the subject; it cannot logically do both. In the following sentences the first two are correct; the third is incorrect. 
  
  (1) The analyst gave a talk on the defense of Matsu and Quemoy. 
  
  (2) The analyst gave a talk \enquote{The Defense of Matsu and Quemoy} (or, a talk entitled \enquote{The Defense of Matsu and Quemoy}). 
  
  (3) The analyst gave a talk on \enquote{The Defense of Matsu and Quemoy.} 
  
  \item \emph{\enquote{and so forth}} after \enquote{e.g.} In \enquote{e.g., As, B, C, etc.} the \enquote{etc.} contradicts the \enquote{e.g.} (which means \enquote{for example}) just as it contradicts \enquote{such as} in another exammple. Omit \enquote{etc.} When you give an example be content to give a few and stop.
  
  \item \emph{\enquote{could.}} The questionable, or even silly, use of \enquote{could} exampled below occurs often in articles on sports, probably because sports writers have got into the habit of hedging their predictions (Team A will win tomorrow if it scores more runs than Team B); but it occurs elsewhere too. The point against \enquote{could} in these sentences is this: anything could happen; the question is what will happen or will likely happens and, it is the job of the sports writer and other experts to tell us. Will Mr.\ Nugent wind up in the pro ranks or won't he? (He \emph{could} do any of a number of thing.) Will the Soviet office be given increased importance or will it not? If we don't want to say one way or another, we should use \enquote{may} or let the question alone; we should not put on the seer's mantle and then refuse to seeraay. The Hodges statement is the best example. What kind of a business expert is this who says the economy \emph{could} start upward? Of course it could. And, it could start downward and it could stay put. 
  
  a.\ \enquote{It is entirely possible that the imaginative disciple of wide open football, Tom Nugent of Maryland could wind up in the pro ranks after this season.} (Washington Post, 2 Dee 1960) 
  
  b.\ \enquote{While the office of head of the Soviet state has heretofore been little more than a ceremonial sinecure, it could be given increased importance under the direction of an energetic party careerist.} (OCI, weekly summary, 2 June 1960) 
  
  c.\ \enquote{Secretary of Commerce Luther H. Hodges said today \enquote{I think we've hit the bottom} of the recession, and the economy could start to move upwardo} (NY Times, 13 March 1961) 
  
  \item \emph{\enquote{latter.}} First, there must be two terms and only two and the terms must be parallel; second, do not use \enquote{latter} when you can state the second term in a word or two 
  
  a.\ \enquote{Western public opinion does not clearly understand the importance of Yugoslav revisionism. The latter is preparing the ideological base for\dots.} \censor{some other stuff} Only one term since \enquote{western public opinion'} is not parallel with \enquote{Yugoslav revisionism.} Here instead of \enquote{the latter} we must repeat \enquote{Yugoslav revisionism} or say \enquote{this revisionism.} 
  
  b.\ \enquote{The Chinese communists as, pointed Liu Shao-ch'i and Chou En-lai; the latter was\dots.} This is correct but better to use \enquote{Chou} instead of \enquote{the latter.} 
  
  \item \emph{\enquote{Like.}} When the items that follow \enquote{Like} are the only examples, items, or almost the only ones, \enquote{like} is inept. 
  
  a.\ \enquote{Revolution has succeeded, even when faced by stronger forces, only in countries like China, Vietnam, and Cuba, where the peasants had been mobilized by the Party} omit \enquote{in countries like} and write \enquote{only in china\dots.} 
  
  \item \emph{\enquote{may.}} The trouble with \enquote{may} in the sentences below is a word that precedes it: \enquote{suspects,} \enquote{signs,} \enquote{believed.} With those three words \enquote{may} is redundant; they have already expressed doubt. So the sentences should read that Sarit suspects some of his underlings are plotting again, that there are signs the economy will begin, and that it is believed (here is the passive voice, of course) he will retire. If we want to use \enquote{may} then we must omit other words of doubt and say simply that once the agency is installed in its new quarters Mr.\ Dulles may retire.
  
  a.\ \enquote{Sarit apparently suspects that some of his underlings in the ruling military group may be plotting again.} (OCI, weekly summary, 30 June 1960) 
  
  b.\ \enquote{\dots there are signs the U.S. economy may begin shaking off the recession next month\dots} (Secretary of Labor Arthur Goldberg in the Washington Post, 9 March 1961) 
  
  c.\ \enquote{It is believed that once the agency is installed in the new quarters he may retire.} (NY Times, 5 June 1961) 
  
  \item \emph{\enquote{(largest) of any.}} This is common but nonsensical. The sensible phrases are \enquote{largest of all} (\enquote{the largest apple of all the apples in the orchard}) and \enquote{better than any other.} The \enquote{other} must be expressed since a thing cannot be better than itself. 
  
  \item \emph{\enquote{respectively}} There must be two or more pairs. (See Fowler, p. 500)
  
  a.\ \enquote{The debriefing of X occurred on 28 December 1954 and 20 January 1955 respectively.} Incorrect; omit \enquote{respectively.} 
  
  b.\ \enquote{The debriefing of X and Y occurred on 28 December and 29 December, respectively.} Correct; X on 28 December and Y on 29 December. 
  
  \item \emph{\enquote{such as}} This phrase means examples are to follow (\enquote{X speaks all kinds of Chinese dialects, such as Cantonese and Fukienese}) but it occurs often where \enquote{including} is called for or where the examples are not examples but the only items. 
  
  a.\ \enquote{\dots speeches were translated simultaneously into thirteen languages such as English, Russian, Spanish, French, German, Chinese, Polish, Rumanian, Arabic and Indonesian} (Amcongen Madras Desp No. 202, 15 Dec 60). Use \enquote{including.}
  
  b.\ \enquote{JCP active in many front groups and has achieved dominant influence in such organization as Gensuikyo, Zengakuren, Japan-Soviet Society\dots. etc.} (Amemb Tokyo Tel No. 1654, 8 December 1961).\ \enquote{Such as} implies that the JCP is dominant in organizations other than those named, and names the ones it does as examples. But are not the ones named either all the organizations or the main ones? The \enquote{etc.} furthermore, destroys the \enquote{such as} (if we give examples; we do not name all but a few examples; to conclude with \enquote{etc.} is to violate what we started out to do-give examples). Better to write \enquote{dominant influence in A, B, C. and D.} We may add \enquote{and some other organizations.}
  
  \item \emph{\enquote{that}} and \emph{\enquote{which}} The only distinction between these two relative pronouns is to use \enquote{that} with restrictive clauses and \enquote{which} with non-restrictive. Commonly the two words are interchangeable. (See Fowler, who uses \enquote{defining relative clause} for \enquote{restrictive relative clause,} page 635.) 
  
  \item \emph{\enquote{would}} This word, belonging to the subjunctive mood, is conditional; there is an \enquote{if} clause involved though this clause is usually understood rather than expressed. But we use \enquote{would} when we do not mean condition, when our meaning is indicative. To test this, express the \enquote{if} clause that this use of \enquote{would} usually suppresses. In the first sentence, below, Senator Humphrey is saying \enquote{I would not want to receive the vote\dots if anyone should bring up this issue} but he means that he does not want to receive such votes whether anyone brings up the issue or not. His use of the conditional is his way of speaking, an indirect, a courteous way. And in the second sentence Senator Kennedy means that it is appropriate, not that it would be appropriate if someone should ask him. In the third sentence OCI will welcome comment but modestly says she would welcome it if she should be allowed to express her desires. In the fourth sentence there is no sense at all to \enquote{would appear}-- a, good chance either appears or it does not; and the \enquote{may define} after \enquote{appear} is silly. The correct sentence is \enquote{There appears to be a good chance that Qasim will define\dots.} These uses of \enquote{would} result from some feeling of modesty or courtesy or some fear of risk, of asserting oneself (caution again). They do not belong in simple, straightforward writing. 
  
  a.\ \enquote{I would not want to receive the vote of any American because my opponent or opponents worship in a particular church whatever that church may be} (NY Times, 22 April 1960, Senator Hubert H. Humphrey) 
  
  b.\ \enquote{I have decided, in view of current press reports, that it would be appropriate to speak with you today about what has widely been called the \enquote{religious issuer in American politics}} (Dittoo Senator John F. Kennedy) 
  
  c.\ \enquote{The Sino-Soviet Studies Group would welcome comment on this paper.} (OCI, \enquote{Mao Tse-tung and Historical Materialism,} 10 April 1961, preface) 
  
  d.\ \enquote{There would appear to be a good chance that Qasim may so define the conditions of party activity as to restrict it considerably.}
\end{enumerate}
  
\chapter{A Program for Improvemant}

The trouble with CIA writing is much the same trouble that there is with other governmental writing and with writing generally in the United States, All over the United States parents, graduate schools, and business houses complain about the inferior writing of high-school and college students, CIA.ts problem is unique only in the relation of our profession to our writing but this relation is important, for it lends itself to bad writing and if we do not write well we risk losing valuable information and wasting dangerous efforts CIA writing is wordy and it is full of jargons, of would.-be professional languages, of cliches; it is even opaque, It is not clear, simple, and terse. Grammatical correctness, punctuation, and capitalization are of less concern, although when we put all of these things together we got writing that has no distinction at all save a few bad ones.

There are many ways to improve CIA writing but any thoroughgoing way must be based upon two premises. One is that there will be no sizable and lasting improvement until supervisors refuse to accept bad writing. This is the single most significant fact about improving CIA writing: supervisors accept bad writing so often that they encourage writers to do again what they have done badly before. It is only in a few things, especially those written for the director's office, that care, exceptional care, is given to the writing. Some supervisors and some writers assert that there is nothing one can do about CIA writing because of the squeeze on GIP, writer-so. There is some squeezing, of course, but, with the exception of a cable now and then that must be written without almost any rethinking or rewriting, most CIA writing has the time for some revision. And there is less speed necessary in DDS, than in either DDI or DDP and yet DDS writing is likely the worst of all. The instructions DDS issues give the reader the impression that DDS believes that it is not responsible for the thick paste of jargon, cliches, and business cant because the instructions stay inside CIA and fall into the category of business matters, which fall outside the practices of good English. The supervisors in DDS, therefore, have a harder job to do than those in DDI and DDP to educate both themselves and others, Despite this the problems in DDS are the same in kind as the problems in DDI and DDP, and if we are going to do something about improving CIA writing we should improve it throughout the agency, make everyone aware of it, and the supervisors first of all. A writer may know that he is not writing well but, if his supervisor accepts that he has written, he will not make himself take the pains to write better. Some supervisors know good writing when they see it and others do not. If we encourage those who do to insist on good writing, and if we encourage those who do not to find out what good writing is, we shall take a long step forward. 

The second premise is that good writing requires constant attention. Almost all of us recognize this. It is easy to become slipshod. It is easy to forget what we learned in high school or college. It is easy to attribute our careless writing to emergencies and to the kind of work we do. And yet, on the other hand, the occasional course, lecture, or workshop is not going to accomplish much. These are shots in the arm that will jerk the student into attention for a few weeks or a few months, but when the stimulant wears off or the student returns to supervisors who do not enforce the same standards, the student will fall back into his old ways. We know from the accounts of professional writers that writing is often painful. If these men who earn their living by writing find writing laborious, we can scarcely expect to turn out a well-written dispatch, directive, or study without some pains ourselves. This second premise says that our painstaking must occur over and over again, that it is not a matteronly of a seminar, an essay, or a lecture. 

If we accept these premises, what do we do next? It seems to the writer that we have our choice between something thoroughgoing and something piecemeal. The importance of writing to our job of gathering, reporting, and analyzing intelligence information justifies a thoroughgoing effort. So too does the importance of writing to pride in our work. The following is a plan for such an effort. Put a person in the inspector general's office or the director's office in charge. Make this person responsible for the inspiration and guidance throughout the agency. Put a similar person in each of the three main parts of the agency. Make each of these persons responsible under his deputy director for the writing in his part of the agency and responsible under the person in the central office for carrying out the principles good for all three parts. Request these three persons each to prepare a manual of style for his part and then let the four decide whether to put out separate manuals or just one. (There is much to say about a manual of style and what it should try to accomplisho) Expect OTR to do much as it is now doing in developing and giving courses at the request of the different parts of the agency, and expect it to help in writing any manual of style; and strengthen OTR by the help of the four personae Request OM to prepare a course in the principles of good writing for all employees of the grade GS-14 or over. (There will be other courses; the idea here is to reach the supervisors and writers in the highest grades.) Request OTR to develop a writing test for all or certain newcomers to the agency and for retesting them from time to time. 

These proposals do not intend that the persons mentioned above give their major attention to the mechanics of good writing, to grammar, the length of subjects or titles, spelling, punctuation, capitalization, and so forth. These proposals intend that they give their major attention to the major problems. These major problems are words and the number of words. These problems are concerned with cutting a fifty-page study or directive down to fifteen pages and with writing it in simple and clear English. How much of the work of the four persons will be prescriptive and how much of it will be permissive will depend upon them, and how things work out. The emphasis should be upon inspiration and guidance and not upon regulations. Persons will write well not simply when they take care to follow rules that someone has laid down for them, but when they are encouraged to fulfill what they have learned and respected. Here again the supervisor plays the critical part. Although there is bound to be a certain amount of rather scholastic debate over the agreement of subject and verb, capitalization, and like subjects, the major interest of the supervisor will be in the words and the thought, whether the paper before him is written carefully and even forcefully and in the fewest words possible. If the supervisor looks for and demands good writing and if there are clear standards in every oness mind and if there is constant, almost continuous, interest throughout the agency in good writing, CIA writing will improve. It will improve to the extent that there is a clear and forceful program to improve it.
\backmatter{}
\newpage
\thispagestyle{empty}
\
\end{document}